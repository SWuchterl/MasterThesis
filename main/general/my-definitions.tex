% units and symbols
% most of them are defined in ptdr-definitions.tex, the ones that are missing there are defined here
\newcommand{\pb}{\ensuremath{\,\text{pb}}\xspace}
\newcommand{\ns}{\ensuremath{\,\text{ns}}\xspace}
\newcommand{\K}{\ensuremath{\,\text{K}}\xspace}
\newcommand{\barn}{\ensuremath{\,\text{b}}\xspace}
\newcommand{\MV}{\ensuremath{\,\text{MV}}\xspace}
\newcommand{\MVm}{\ensuremath{\,\text{MV\hspace{-0.16em}/\hspace{-0.08em}m}}\xspace}
\newcommand{\MHz}{\ensuremath{\,\text{MHz}}\xspace}
\newcommand{\kHz}{\ensuremath{\,\text{kHz}}\xspace}
\newcommand{\T}{\ensuremath{\,\text{T}}\xspace}
\newcommand{\mrad}{\ensuremath{\,\text{mrad}}\xspace}
\newcommand{\mt}{\ensuremath{m_{\mathrm{T}}}\xspace}
\newcommand{\metSlash}{\ensuremath{{\not\mathrel{E}}_\mathrm{T}}} % alternative version to \ETslash, w/o spacing problem
\newcommand{\mtTwo}{\ensuremath{M_{\mathrm{T2}}}\xspace}
% period and comma after formulas with some extra spacing


% \providecommand{\PSGc}{\ensuremath{\widetilde{\chi}}\xspace} % neutralino
% \providecommand{\PSGcz}{\ensuremath{\widetilde{\chi}^0}\xspace} % neutralino with superscript 0
\providecommand{\neutralinoOne}{\ensuremath{\widetilde{\chi}^{0}_{1}}\xspace} % neutralino
\providecommand{\PSGcmDo}{\ensuremath{\widetilde{\chi}^{-}_{1}}\xspace} % neutralino
\providecommand{\neutralinoTwo}{\ensuremath{\widetilde{\chi}^{0}_{2}}\xspace} % neutralino
\providecommand{\neutralinoThree}{\ensuremath{\widetilde{\chi}^{0}_{3}}\xspace} % neutralino
\providecommand{\neutralinoFour}{\ensuremath{\widetilde{\chi}^{0}_{4}}\xspace} % neutralino
% \providecommand{\PSGcpm}{\ensuremath{\widetilde{\chi}^\pm}\xspace} % neutralino
\providecommand{\charginoOne}{\ensuremath{\widetilde{\chi}^\pm_{1}}\xspace} % neutralino
\providecommand{\charginoOneBar}{\ensuremath{\widetilde{\chi}^\mp_{1}}\xspace} % neutralino
\providecommand{\charginoTwo}{\ensuremath{\widetilde{\chi}^{+}_{2}}\xspace} % neutralino
\providecommand{\gravitino}{\ensuremath{\widetilde{G}\xspace}} % neutralino


\newcommand{\paf}{\ .}
\newcommand{\caf}{\ ,}
\newcommand{\waf}[1]{\ \text{#1}}

% references
\AtBeginDocument{ % hyperref redefines \ref at beginning of the document
 \let\oldref\ref
 %%% PubCom recommendations are given as comments
 %%% Currently, I do not follow PubCom recommendations
 \newcommand{\refChap}[1]{\hyperref[#1]{Chapter~\oldref*{#1}}}		% 'Chapter 3'
 \newcommand{\refSec} [1]{\hyperref[#1]{Section~\oldref*{#1}}}		% 'Section 3'
 \newcommand{\refApp} [1]{\hyperref[#1]{Appendix~\oldref*{#1}}}		% 'Appendix B'
 \newcommand{\refFig} [1]{\hyperref[#1]{Figure~\oldref*{#1}}}		% 'Fig. 3'; at beginning of sentence 'Figure 3'
 \newcommand{\refTab} [1]{\hyperref[#1]{Table~\oldref*{#1}}}			% 'Table 3'
 \newcommand{\refEq}  [1]{\hyperref[#1]{Equation~(\oldref*{#1})}}% 'Eq. (3)'; at beginning of sentence 'Equation (3)'
}

%% singlets and doublets (for SM table)
\newcommand{\doublet}[2]{$\begin{pmatrix} #1 \\ #2 \end{pmatrix}_{\mathrm{L}}$}
\newcommand{\lsinglet}[1]{$\begin{array}{c} #1_\mathrm{R}^\mathrm{-}\end{array}$}
\newcommand{\qsinglet}[2]{$\begin{array}{c} #1_\mathrm{R} \\ #2_\mathrm{R}\end{array}$}
\newcommand{\doubarrc}[2]{$\begin{array}{c} #1 \\ #2  \end{array}$}
\newcommand{\doubarrl}[2]{$\hspace{-2ex}\begin{array}{l} #1 \\ #2  \end{array}$}
\newcommand{\doubarrr}[2]{$\begin{array}{r} #1 \\ #2  \end{array}\hspace{-2ex}$}
\newcommand{\singarrl}[1]{$\hspace{-2ex}\begin{array}{l} #1  \end{array}$}
\newcommand{\singarrr}[1]{$\begin{array}{r} #1  \end{array}\hspace{-2ex}$}

% defined to be equal symbol :=
\newcommand*{\defeq}{\mathrel{\vcenter{\baselineskip0.5ex \lineskiplimit0pt
   \hbox{\scriptsize.}\hbox{\scriptsize.}}}%
 =}

% ########## Hyphenations ##########
% \hyphenation{ex-am-ple}
