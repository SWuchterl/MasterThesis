\chapter{Introduction}\label{chap:intro}

% One of the main questions physicists have been asking for ages is, what matter is made of, and how this is connected to the description of the fundamental forces.\\
The understanding of the world around us is one of the main reasons for science. Driven by the urge for knowledge and striving to combine all observed phenomena in a unified description gave rise to more and more successful and comprehensive models in all fields of physics.\\
The state-of-the-art understanding in particles physics is characterized by the standard model (SM), which describes all observations in high energy particle physics successfully. Essential parameters of the SM have been studied at particle accelerators for years with very high precision. It provides a combined description of three of the four fundamental forces and a classification of all building blocks of matter. The most recent remarkable success of the standard model is the observation of a Higgs boson in 2012~\cite{HiggsCMS,HiggsATLAS,HiggsCombined}. Since then, the properties and couplings of the Higgs boson have been studied also with great interest and effort.\\\\
Nevertheless, the SM is not capable of explaining various physical observations, such as the existence of neutrino masses or the origin of dark matter. Several theoretical extensions to the SM could supply answers to the different shortcomings. Driven by the attractive concept of symmetry~\cite{Noether} in theory, one possible solution is supersymmetry (SUSY). SUSY postulates partner particles to each particle of the SM, introducing a symmetry between fermions and bosons, although the supersymmetric partners must be much heavier in mass compared to their SM equivalents.\\\\
Bigger and more complex, and such more expensive, experiments are needed to study elementary particles at the high energy frontier. The currently largest and most powerful particle accelerator is the Large Hadron Collider (LHC) at CERN near Geneva. Here, billions of protons are accelerated to almost the speed of light and are collided every second. The large amount of data arising thereby is recorded with the Compact Muon Solenoid (CMS) detector located at one of the four collision points of the LHC. The data sample being analyzed in this thesis consists of events recorded in 2016 based on proton-proton collisions with a center-of-mass energy of $13\TeV$. CMS shows an excellent performance with regard to event reconstruction and particle identification, based on the combined reconstruction from all CMS subdetectors, such as the tracking system, the electromagnetic and hadronic calorimeters, and the muon system.\\\\\\
Supersymmetric particles are expected to have masses that might be accessible at the LHC. In order to difference between the masses of the SM particles and their partners, a symmetry breaking mechanism needs to be introduced. One scenario is gauge-mediated supersymmetry breaking (GMSB), where the second lightest supersymmetric particle is expected to decay to SM bosons, such as the Z boson and the photon, and the lightest supersymmetric particle (LSP) which is the gravitino ($\gravitino$). The LSP is supposed to be undetectable, thus leading to a imbalance in the measured transverse momentum sum, called missing transverse momentum. These theoretical models are mainly motivating the presented search. \\
Searches for SUSY with photons in the final states have been carried out using the 2016 data set~\cite{PhotonHT,PhotonMet,PhotonBJet,PhotonLepton}, but in this thesis a new independent final state is investigated, complementing the existing searches. By targeting events with photons, a Z boson decaying to charged leptons, and missing transverse momentum, the search is sensitive to various production channels for SUSY particles, and is able to study also phase space regions, where not so much energy is transferred to the final state particles, due to low requirements on the lepton and photon momenta.\\
The analyzed data set is based on events where at least two same-flavor opposite-charge leptons are reconstructed, being a part of the total 2016 data set corresponding to a total integrated luminosity of $35.9\fbinv$.\\
Standard model processes leading to the same final state are considered as background throughout this thesis, and they are estimated using an approach based on Monte-Carlo simulated events that are normalized to data in various control regions. In the signal sensitive region, a counting experiment is performed, comparing SM prediction with supersymmetric signal expectations. The considered SUSY signals are based on three different signal scenarios, all established within the General Gauge Mediation (GGM) framework.\\
This analysis has been endorsed by the CMS collaboration, and is documented in detail in an official CMS analysis note~\cite{MyAN}.\\\\
This thesis is structured as follows.
Firstly, a theoretical introduction to the SM and supersymmetry is given in \refSec{chap:introduction}, where SUSY in general is discussed, before in particular gauge-mediated supersymmetry breaking is explained, and the relevant signal models are introduced. Afterwards, the Large Hadron Collider and the CMS experiment are presented in \refSec{chap:experiment}. In \refSec{chap:reco} the basic event processing, including reconstruction and triggering, together with event simulation is explained, and properties of the used data sets are introduced. Moreover, the identification principle of physical objects, the definition of important high level variables, as well as the event selection and trigger efficiency measurement are discussed. The basic analysis strategy and background prediction is explained in detail in \refSec{chap:analysis}, while the definition of the control, validation and signal regions are given. In addition, the estimation of the background and validation of the background estimation are discussed. The determination of systematic uncertainties is also presented. Final results are discussed in \refSec{chap:results} together with their interpretation in the context of various signal models. In \refSec{chap:conclusion}, the thesis is briefly summed up, and conclusions are made.
