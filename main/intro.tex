\chapter{Introduction}\label{chap:intro}

One of the main questions physicists have been asking for ages is, what matter is made of, and how this is connected to the description of the fundamental forces.\\
The current state-of-the-art understanding is characterized by the standard model of particle physics (SM), which describes all observations in high energy particle physics successfully. Essential parameters of the SM have been studied at particle accelerators for years with very high precision. It provides a combined description of three of the four fundamental forces and a classification of all building blocks of matter. The most recent remarkable success of the standard model was the observation of the postulated Higgs boson in 2012~\cite{HiggsCMS,HiggsATLAS,HiggsCombined}. Since then, the Higgs sector has been studied also with great interest and effort.\\\\
Nevertheless, the standard model is not capable of explaining various physical observations, such as the existence of neutrino masses or the origin of dark matter. Several theoretical extensions to the standard model could supply answers to the different shortcomings of the standard model. Driven by the concept of symmetry being present in physics for decades~\cite{Noether}, one attractive solution is supersymmetry. Supersymmetry postulates partner particles to each particle of the standard model differing in their spin, although the supersymmetric partners must be much heavier in mass compared to their standard model equivalents.\\\\
Bigger and more complex, and such more expensive, experiments are needed to study elementary particles at the high energy frontier. The currently largest and most powerful particle accelerator is the Large Hadron Collider (LHC) at CERN near Geneva. Here, billions of protons are accelerated to the speed of light and are collided every second. The large amount of data arising thereby is recorded with the Compact Muon Solenoid (CMS) detector located at one of the four collision points of the LHC. The data sample being analyzed in this thesis consists of events recorded in 2016 based on proton-proton collisions with a center of mass energy of $13\TeV$.\\\\
Supersymmetric particles are expected to have masses that are accessible at the LHC. In order to obtain a deviation between the masses of the standard model particles and their partners, a symmetry breaking mechanism needs to be introduced. One scenario is gauge-mediated supersymmetry breaking (GMSB), where the second lightest supersymmetric particle is expected to decay to standard model bosons, such as the Z boson and the photon, and the lightest supersymmetric particle (LSP). The LSP is supposed to be undetectable, thus leading to a imbalance in the measured transverse momentum sum, called missing transverse momentum. Hence, by making use of the excellent performance of the combined reconstruction from all CMS subdetectors, such as the tracking system, the electromagnetic and hadronic calorimeters, and the muon system, these theoretical models motivate the presented search.\\
Searches for SUSY with photons in the final states have been carried out~\cite{PhotonHT,PhotonMet,PhotonBJet,PhotonLepton}, but in this thesis a new independent final state is investigated, complementing the existing searches. By targeting events with photons, a Z boson decaying to charged leptons, and missing transverse momentum, the search is sensitive to various production channels for SUSY particles, and is able to study also phase space regions, where the not so much energy is transferred to the final state particles, due to low requirements on the lepton and photon momenta.\\
The analyzed data set is based on events where at least two same-flavor  opposite-charge leptons are reconstructed, being a part of the total 2016 data set corresponding to a total integrated luminosity of $35.9\fbinv$.\\
Standard model processes leading to the same final state are considered as background throughout this thesis, and they are estimated using a approach based on Monte Carlo simulated events, that are normalized to measured data in various control regions. In the sensitive signal region, a counting experiment is performed, comparing standard model prediction with supersymmetric signal expectations. The considered SUSY signals are based on three different theoretical models, all established within the General Gauge Mediation (GGM) framework.\\
This analysis has been endorsed by the CMS collaboration, and is documented in detail in~\cite{MyAN}.\\\\
This thesis is structured as follows.
Firstly, a theoretical introduction to the SM and supersymmetry is given in \refSec{chap:introduction}, where SUSY in general is discussed, before in particular gauge-mediated supersymmetry breaking is explained, and the relevant signal models are introduced. Afterwards, the Large Hadron Collider and the CMS experiment as the experimental setup are presented in \refSec{chap:experiment}. In \refSec{chap:reco} the basic event processing, including reconstruction and triggering, together with event simulation is explained, and properties of the used data sets are introduced. Moreover, the identification principle of physical objects, the definition of important high level variables, as well as the event selection and trigger efficiency measurement are discussed. The basic analysis strategy and background prediction is explained in detail in \refSec{chap:analysis}, while the definition of the control, validation and signal regions are given. In addition, the estimation of the background and validation of the background estimation are discussed. The determination of systematic uncertainties is also presented. Final results are discussed in \refSec{chap:results} together with their interpretation in various signal models. In the end, in \refSec{chap:conclusion}, the thesis is briefly summed up, and conclusions are made.
