\chapter{Summary \& Conclusion}\label{chap:conclusion}

A search for supersymmetry (SUSY) in final states with missing transverse momentum $\ptmiss$, a Z boson decaying to charged leptons ($\PZ\to\Pe\Pe/\Pgm\PGm$), and photons based on a data set of proton-proton collisions with a center-of-mass energy of $13\TeV$ recorded with CMS in 2016 has been presented. The data set corresponds to an integrated luminosity of $35.9\fbinv$, and the relevant events are selected using same flavor dilepton triggers.\\\\
Events with photons, Z bosons, and $\ptmiss$ are expected in gauge-mediated supersymmetry breaking (GMSB) scenarios, where the next-to-lightest supersymmetric particle (NLSP) is bino- or wino-like, and the NLSP decays to these standard model (SM) bosons and the lightest supersymmetric particle (LSP), which is the gravitino $\gravitino$. The gravitino is predicted to be stable in R-parity conserving SUSY scenarios and leaves the detector undetected, thus giving rise to a significant amount of $\ptmiss$ in the event.\\\\
The search was developed such, that it is complementary to other CMS GMSB SUSY searches targeting different scenarios including photons. Therefore, this is the first CMS SUSY analysis investigating explicitly $\PZ(\to\ell\ell)+\PGg$ events. In addition, the analysis is designed to be sensitive to different production channels of gauginos, including strong and electroweak production channels. By restricting the search to events including three final state particles and $\ptmiss$, the requirements on the final states momenta can be low. The usage of the stransverse mass $\mtTwo$ reconstructed from the Z boson candidate, the photon, and the missing transverse momentum, enables good discrimination between background and SUSY signals by estimating the NLSP mass.\\\\
The main background contributions for the search are $\ttbar(\PGg)$, DY/$\PZ(\PGg)$, $\PW\PZ$, and $\PZ\PZ$ production, while $\ttbar(\PGg)$ is by far the most important background, making up $70-80\%$ of the total SM background in the signal region. Each of those backgrounds is estimated with an approach based on Monte-Carlo simulation, where each background is normalized to the observation in a dedicated control region. Data and prediction are observed to agree in each control region, and further studies such as $\chi^2$-fits are performed to investigate the modeling of different observables within the simulation.\\
The final background prediction is validated in a validation region, which is constructed as a kinematic sideband region to the signal region, by requiring the events to fail one of the two additional signal region requirements. The agreement between data and prediction in the validation region is also good, indicating that the background estimations is stable.\\\\
In the signal region, a counting experiment is performed comparing predicted and observed event yields in each of the bins of the $\ptmiss$ distribution. No significant deviation is observed.\\\\
Upper cross section and exclusion limits are calculated at $95\%$ confidence level for three different signal models. The results are interpreted within an electroweak model based on the General Gauge Mediation framework, excluding bino and wino masses up to $400-560\GeV$. Additional simplified models are used for interpretation, excluding NLSP masses up to $600\GeV$ in case of electroweak gaugino production. In cases of gluino pair production, gluino masses lower than $1.4\TeV$ are excluded for low mass differences between the gluino and the NLSP, decreasing for larger mass differences.\\\\
The results presented throughout this thesis are documented in~\cite{MyAN}. The exclusion limits are comparative to different CMS SUSY searches~\cite{PhotonMet} in the low bino and wino mass regions of the GMSB model, but are not as sensitive as other analyses for the simplified models discussed above due to the low $\PZ\to\ell\ell$ branching fraction. Since the analysis investigates a new final state with respect to all other CMS GMSB SUSY searches~\cite{PhotonMet,PhotonHT,PhotonBJet}, it can be useful for a future statistical combination of analyses, which is under development~\cite{Danilo} at the time of writing. With the full Run II data recorded at CMS in the years 2015-2018, a reperformance of this search on the much larger data set would be interesting, as the larger data set enables possibilities to optimize the background prediction, especially the search region definitions, and thus compensates the small branching fraction of the leptonic Z boson decays.
