\chapter{Conclusion}\label{chap:conclusion}

A search for gauge-mediated supersymmetry in final states with missing transverse momentum $\ptmiss$, a Z boson decaying to charged leptons ($\PZ\to\Pgm\PGm/\Pe\Pe$), and photons based on events of proton-proton collisions with a center-of-mass energy of $13\TeV$ recorded with CMS in 2016 has been presented. The data set corresponds to an integrated luminosity of $35.9\fbinv$, and the relevant events are selected using same flavor dilepton triggers.\\\\
Events with photons, Z bosons, and $\ptmiss$ are expected in gauge-mediated supersymmetry breaking scenarios, where the next to lightest supersymmetric particle has bino and wino components in the mixing, and the NLSP decays to these SM bosons and the lightest supersymmetric particle, the gravitino $\gravitino$. The gravitino is predicted to be stable in R-parity conserving SUSY scenarios and leaves the detector undetected, thus giving rise to a significant amount of $\ptmiss$ in the event.\\
The search was developed such, that is complementary to other CMS GMSB SUSY searches targeting different final state scenarios including photons. Therefore, this is the first CMS SUSY analysis investigating $\PZ(\to\ell\ell)+\PGg$ events. In addition, the analysis is designed to be sensitive to different production channels of gauginos, including strong and electroweak production channels. By restricting to events including already three final state particles and $\ptmiss$, not many additional selection criteria are needed to reduce standard model background contributions. The usage of the stransverse mass $\mtTwo$ reconstructed from the Z boson candidate, the photon, and the missing transverse momentum, enables good discrimination between background and SUSY signals by estimating the NLSP mass.\\\\
Main background contributions for the search are $\ttbar(\PGg)$, DY/$\PZ(PGg)$, $\PW\PZ$, and $\PZPZ$ production, while $\ttbar(\PGg)$ is by far the most important with a relative amount of $\approx70-/80\%$. Each of those backgrounds is estimated with a data driven approach based on Monte-Carlo simulation, where each background is normalized to the observation in a dedicated control region. The agreement in each control region is good, and further studies such as $\chi^2$-fits are performed to investigate the modeling of different observables within the simulation.\\
The final background prediction is validated in a particular validation region, which is constructed as a sideband region to the signal region, by requiring the events to fail one of the two signal region requirements. The agreement in the validation region is also very good, indicating that the background estimations behaves properly.\\\\
In the signal region a counting experiment is performed comparing predicted and observed yields in each of the bins in the $\ptmiss$ distribution. No significant deviation is observed, though in on of the bins a $1.3\sigma$ upward deviation is present.\\
Upper cross section and exclusion limits are calculated at $95\%$ confidence level for three different signal models. If the results are interpreted within a electroweak model based on the General Gauge Mediation framework, bino and wino masses up to $400-560\GeV$ can be excluded. Additional simplified models are used for interpretation, NLSP neutralino masses up to $600\GeV$ can be excluded in case of electroweak gaugino production. In cases of gluino pairproduction, gluino masses lower than $1.4\TeV$ can be excluded.\\\\
The results presented throughout this thesis are the same as documented in~\cite{MyAN}. The exclusion limits are comparative to different CMS SUSY searches~\cite{PhotonMet} in the low bino and wino mass regions of the GMSB model, but are not that sensitive compared to other analyses in interpreting the simplified models discussed above due to the lower $\PZ\to\ell\ell$ branching fraction. Hence, the mass and cross section exclusion limits are not as high as of the other searches. Since the analysis investigates a new final state with respect to all other CMS GMSB SUSY searches~\cite{PhotonMet,PhotonHT,PhotonBJet}, it might be useful for a future statistical combination of analyses, like it is under development~\cite{Danilo} at the time of writing. With the full Run II data recorded at CMS in the years 2015-2018, a reperformance of this search with a much larger data set would be quite interesting, because more data enables possibilities to optimize the background prediction, and especially the search region definitions.
