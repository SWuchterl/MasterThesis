%%% General settings

% \def\Fileversion$#1: #2 ${\gdef\fileversion{#2}}
\def\Filedate$#1: #2-#3-#4 #5 ${\gdef\filedate{#2/#3/#4}}
\Fileversion$Revision: 466930 $
\Filedate$Date: 2018-06-30 19:31:24 +0200 (Sa, 30 Jun 2018) $
%%%%%%%%%%%%%%%%%%%%%%%%%%%%%%%%%%%%%%%%%%%%%%%%%%%%%%%%%%%%%%%%%%%%
%
%  CMS Common definitions style file
%
%  N.B. use of \newcommand rather than \newcommand means
%       that a definition is ignored if already specified
%
%                                              L. Taylor 18 Feb 2005
%%%%%%%%%%%%%%%%%%%%%%%%%%%%%%%%%%%%%%%%%%%%%%%%%%%%%%%%%%%%%%%%%%%%
\NeedsTeXFormat{LaTeX2e}
\ProvidesPackage{ptdr-definitions}[\filedate\space CMS Additional Macro Definitions (\fileversion)]
\RequirePackage{xspace}
\RequirePackage{amsmath}

% Some shorthand
% turn off italics
\newcommand {\etal}{\mbox{et al.}\xspace} %et al. - no preceding comma
\newcommand {\ie}{\mbox{i.e.}\xspace}     %i.e.
\newcommand {\eg}{\mbox{e.g.}\xspace}     %e.g.
\newcommand {\etc}{\mbox{etc.}\xspace}     %etc.
\newcommand {\vs}{\mbox{\sl vs.}\xspace}      %vs.
\newcommand {\mdash}{\ensuremath{\mathrm{-}}} % for use within formulas
\providecommand {\NA}{\ensuremath{\text{---}}}    % for Not applicable (or available). Needs to be renewcommanded for APS to \cdots

% some terms whose definition we may change
\newcommand {\Lone}{Level-1\xspace} % Level-1 or L1 ?
\newcommand {\Ltwo}{Level-2\xspace}
\newcommand {\Lthree}{Level-3\xspace}

% Some software programs (alphabetized)
\newcommand{\ACERMC} {\textsc{AcerMC}\xspace}
\newcommand{\ALPGEN} {{\textsc{alpgen}}\xspace}
\newcommand{\BLACKHAT} {{\textsc{BlackHat}}\xspace}
\newcommand{\CALCHEP} {{\textsc{CalcHEP}}\xspace}
\newcommand{\CHARYBDIS} {{\textsc{charybdis}}\xspace}
\newcommand{\CMKIN} {\textsc{cmkin}\xspace}
\newcommand{\CMSIM} {{\textsc{cmsim}}\xspace}
\newcommand{\CMSSW} {{\textsc{cmssw}}\xspace}
\newcommand{\COBRA} {{\textsc{cobra}}\xspace}
\newcommand{\COCOA} {{\textsc{cocoa}}\xspace}
\newcommand{\COMPHEP} {\textsc{CompHEP}\xspace}
\newcommand{\EVTGEN} {{\textsc{evtgen}}\xspace}
\newcommand{\FAMOS} {{\textsc{famos}}\xspace}
\newcommand{\FASTJET} {{\textsc{FastJet}}\xspace}
\newcommand{\FEWZ} {{\textsc{fewz}}\xspace}
\newcommand{\GARCON} {\textsc{garcon}\xspace}
\newcommand{\GARFIELD} {{\textsc{garfield}}\xspace}
\newcommand{\GEANE} {{\textsc{geane}}\xspace}
\newcommand{\GEANTfour} {{\textsc{Geant4}}\xspace}
\newcommand{\GEANTthree} {{\textsc{geant3}}\xspace}
\newcommand{\GEANT} {{\textsc{geant}}\xspace}
\newcommand{\HDECAY} {\textsc{hdecay}\xspace}
\newcommand{\HERWIG} {{\textsc{herwig}}\xspace}
\newcommand{\HERWIGpp} {{\textsc{herwig++}}\xspace}
\newcommand{\POWHEG} {{\textsc{powheg}}\xspace}
\newcommand{\HIGLU} {{\textsc{higlu}}\xspace}
\newcommand{\HIJING} {{\textsc{hijing}}\xspace}
\newcommand{\HYDJET} {{\textsc{hydjet}}\xspace}
\newcommand{\IGUANA} {\textsc{iguana}\xspace}
\newcommand{\ISAJET} {{\textsc{isajet}}\xspace}
\newcommand{\ISAPYTHIA} {{\textsc{isapythia}}\xspace}
\newcommand{\ISASUGRA} {{\textsc{isasugra}}\xspace}
\newcommand{\ISASUSY} {{\textsc{isasusy}}\xspace}
\newcommand{\ISAWIG} {{\textsc{isawig}}\xspace}
\newcommand{\MADGRAPH} {\textsc{MadGraph}\xspace}
\newcommand{\MCATNLO} {\textsc{mc@nlo}\xspace}
\newcommand{\MCFM} {\textsc{mcfm}\xspace}
\newcommand{\MILLEPEDE} {{\textsc{millepede}}\xspace}
\newcommand{\ORCA} {{\textsc{orca}}\xspace}
\newcommand{\OSCAR} {{\textsc{oscar}}\xspace}
\newcommand{\PHOTOS} {\textsc{photos}\xspace}
\newcommand{\PROSPINO} {\textsc{prospino}\xspace}
\newcommand{\PYTHIA} {{\textsc{pythia}}\xspace}
\newcommand{\SHERPA} {{\textsc{sherpa}}\xspace}
\newcommand{\TAUOLA} {\textsc{tauola}\xspace}
\newcommand{\TOPREX} {\textsc{TopReX}\xspace}
\newcommand{\XDAQ} {{\textsc{xdaq}}\xspace}
\newcommand{\MGvATNLO}{\MADGRAPH{}5\_a\MCATNLO}


%  Experiments
\newcommand {\DZERO}{D0\xspace}     %etc.


% Measurements and units...

\newcommand{\de}{\ensuremath{^\circ}}
\newcommand{\ten}[1]{\ensuremath{\times \text{10}^\text{#1}}}
\newcommand{\unit}[1]{\ensuremath{\text{\,#1}}\xspace}
\newcommand{\mum}{\ensuremath{\,\mu\text{m}}\xspace}
\newcommand{\micron}{\ensuremath{\,\mu\text{m}}\xspace}
\newcommand{\cm}{\ensuremath{\,\text{cm}}\xspace}
\newcommand{\m}{\ensuremath{\,\text{m}}\xspace}
\newcommand{\mm}{\ensuremath{\,\text{mm}}\xspace}
\newcommand{\km}{\ensuremath{\,\text{km}}\xspace}
\newcommand{\mus}{\ensuremath{\,\mu\text{s}}\xspace}
\newcommand{\keV}{\ensuremath{\,\text{ke\hspace{-.08em}V}}\xspace}
\newcommand{\MeV}{\ensuremath{\,\text{Me\hspace{-.08em}V}}\xspace}
\newcommand{\MeVns}{\ensuremath{\text{Me\hspace{-.08em}V}}\xspace} % no leading thinspace
\newcommand{\GeV}{\ensuremath{\,\text{Ge\hspace{-.08em}V}}\xspace}
% \newcommand{\e}{\ensuremath{\,\text{e\xspace}
\newcommand{\GeVns}{\ensuremath{\text{Ge\hspace{-.08em}V}}\xspace} % no leading thinspace
\newcommand{\gev}{\GeV}
\newcommand{\TeV}{\ensuremath{\,\text{Te\hspace{-.08em}V}}\xspace}
\newcommand{\TeVns}{\ensuremath{\text{Te\hspace{-.08em}V}}\xspace} % no leading thinspace
\newcommand{\PeV}{\ensuremath{\,\text{Pe\hspace{-.08em}V}}\xspace}
\newcommand{\keVc}{\ensuremath{{\,\text{ke\hspace{-.08em}V\hspace{-0.16em}/\hspace{-0.08em}}c}}\xspace}
\newcommand{\MeVc}{\ensuremath{{\,\text{Me\hspace{-.08em}V\hspace{-0.16em}/\hspace{-0.08em}}c}}\xspace}
\newcommand{\GeVc}{\ensuremath{{\,\text{Ge\hspace{-.08em}V\hspace{-0.16em}/\hspace{-0.08em}}c}}\xspace}
\newcommand{\GeVcns}{\ensuremath{{\text{Ge\hspace{-.08em}V\hspace{-0.16em}/\hspace{-0.08em}}c}}\xspace} % no leading thinspace
\newcommand{\TeVc}{\ensuremath{{\,\text{Te\hspace{-.08em}V\hspace{-0.16em}/\hspace{-0.08em}}c}}\xspace}
\newcommand{\keVcc}{\ensuremath{{\,\text{ke\hspace{-.08em}V\hspace{-0.16em}/\hspace{-0.08em}}c^\text{2}}}\xspace}
\newcommand{\MeVcc}{\ensuremath{{\,\text{Me\hspace{-.08em}V\hspace{-0.16em}/\hspace{-0.08em}}c^\text{2}}}\xspace}
\newcommand{\GeVcc}{\ensuremath{{\,\text{Ge\hspace{-.08em}V\hspace{-0.16em}/\hspace{-0.08em}}c^\text{2}}}\xspace}
\newcommand{\GeVccns}{\ensuremath{{\text{Ge\hspace{-.08em}V\hspace{-0.16em}/\hspace{-0.08em}}c^\text{2}}}\xspace} % no leading thinspace
\newcommand{\TeVcc}{\ensuremath{{\,\text{Te\hspace{-.08em}V\hspace{-0.16em}/\hspace{-0.08em}}c^\text{2}}}\xspace}

\newcommand{\pbinv} {\mbox{\ensuremath{\,\text{pb}^\text{$-$1}}}\xspace}
\newcommand{\binv} {\mbox{\ensuremath{\,\text{b}^\text{$-$1}}}\xspace}
\newcommand{\fbinv} {\mbox{\ensuremath{\,\text{fb}^\text{$-$1}}}\xspace}
\newcommand{\nbinv} {\mbox{\ensuremath{\,\text{nb}^\text{$-$1}}}\xspace}
\newcommand{\mubinv} {\ensuremath{\,\mu\mathrm{b}^{-1}}\xspace}
\newcommand{\mbinv} {\ensuremath{\,\mathrm{mb}^{-1}}\xspace}
\newcommand{\percms}{\ensuremath{\,\text{cm}^\text{$-$2}\,\text{s}^\text{$-$1}}\xspace}
\newcommand{\lumi}{\ensuremath{\mathcal{L}}\xspace}
\newcommand{\Lumi}{\ensuremath{\mathcal{L}}\xspace}%both upper and lower
%
% Need a convention here:
\newcommand{\LvLow}  {\ensuremath{\mathcal{L}=\text{10}^\text{32}\,\text{cm}^\text{$-$2}\,\text{s}^\text{$-$1}}\xspace}
\newcommand{\LLow}   {\ensuremath{\mathcal{L}=\text{10}^\text{33}\,\text{cm}^\text{$-$2}\,\text{s}^\text{$-$1}}\xspace}
\newcommand{\lowlumi}{\ensuremath{\mathcal{L}=\text{2}\times \text{10}^\text{33}\,\text{cm}^\text{$-$2}\,\text{s}^\text{$-$1}}\xspace}
\newcommand{\LMed}   {\ensuremath{\mathcal{L}=\text{2}\times \text{10}^\text{33}\,\text{cm}^\text{$-$2}\,\text{s}^\text{$-$1}}\xspace}
\newcommand{\LHigh}  {\ensuremath{\mathcal{L}=\text{10}^\text{34}\,\text{cm}^\text{$-$2}\,\text{s}^\text{$-$1}}\xspace}
\newcommand{\hilumi} {\ensuremath{\mathcal{L}=\text{10}^\text{34}\,\text{cm}^\text{$-$2}\,\text{s}^\text{$-$1}}\xspace}

% Physics symbols ...

\newcommand{\PT}{\ensuremath{p_{\mathrm{T}}}\xspace}
\newcommand{\pt}{\ensuremath{p_{\mathrm{T}}}\xspace}
\newcommand{\ET}{\ensuremath{E_{\mathrm{T}}}\xspace}
\newcommand{\HT}{\ensuremath{H_{\mathrm{T}}}\xspace}
\newcommand{\mT}{\ensuremath{m_{\mathrm{T}}}\xspace}
\newcommand{\mTii}{\ensuremath{m_{\mathrm{T2}}}\xspace}
\newcommand{\et}{\ensuremath{E_{\mathrm{T}}}\xspace}
\newcommand{\Em}{\ensuremath{E\hspace{-0.6em}/}\xspace}
\newcommand{\Pm}{\ensuremath{p\hspace{-0.5em}/}\xspace}
\newcommand{\PTm}{\ensuremath{{p}_\mathrm{T}\hspace{-1.02em}/\kern 0.5em}\xspace}
\newcommand{\PTslash}{\PTm}
\newcommand{\ETm}{\ensuremath{E_{\mathrm{T}}^{\text{miss}}}\xspace}
\newcommand{\MET}{\ETm}
\newcommand{\ETmiss}{\ETm}
\newcommand{\ptmiss}{\ensuremath{\pt^\text{miss}}\xspace}
\newcommand{\ETslash}{\ensuremath{E_{\mathrm{T}}\hspace{-1.1em}/\kern0.45em}\xspace}
\newcommand{\VEtmiss}{\ensuremath{{\vec E}_{\mathrm{T}}^{\text{miss}}}\xspace}
\newcommand{\ptvec}{\ensuremath{{\vec p}_{\mathrm{T}}}\xspace}
\newcommand{\ptvecmiss}{\ensuremath{{\vec p}_{\mathrm{T}}^{\kern1pt\text{miss}}}\xspace}
\newcommand{\tauh}{\ensuremath{\PGt_\mathrm{h}}\xspace}
\newcommand{\sqrtsNN}{\ensuremath{\sqrt{\smash[b]{s_{_{\mathrm{NN}}}}}}\xspace}
\newcommand{\mht}{\ensuremath{H_{\mathrm{T}}^{\text{miss}}}\xspace}
\newcommand{\htvecmiss}{\ensuremath{\vec{H}_{\text{T}}^{\text{miss}}}\xspace}

% roman face derivative
\newcommand{\dd}[2]{\ensuremath{\frac{\cmsSymbolFace{d} #1}{\cmsSymbolFace{d} #2}}}
\newcommand{\ddinline}[2]{\ensuremath{\cmsSymbolFace{d} #1/\cmsSymbolFace{d} #2}}
\newcommand{\rd}{\ensuremath{\cmsSymbolFace{d}}}
\newcommand{\re}{\ensuremath{\cmsSymbolFace{e}}}
% absolute value
\newcommand{\abs}[1]{\ensuremath{\lvert #1 \rvert}}
% misc
\newcommand{\CL}{\ensuremath{\text{CL}}\xspace}
\newcommand{\CLs}{\ensuremath{\text{CL}_\text{s}}\xspace}
\newcommand{\CLsb}{\ensuremath{\text{CL}_\text{s+b}}\xspace}



\ifthenelse{\boolean{cms@italic}}{\newcommand{\cmsSymbolFace}[1]{#1}}{\newcommand{\cmsSymbolFace}[1]{\mathrm{#1}}}

% Particle names which track the italic/non-italic face convention
\newcommand{\zp}{\ensuremath{{\cmsSymbolFace{Z}^\prime}}\xspace} % plain Z'
\newcommand{\JPsi}{\ensuremath{{\cmsSymbolFace{J}\hspace{-.08em}/\hspace{-.14em}\psi}}\xspace} % J/Psi (no mass)
\newcommand{\Z}{\ensuremath{\cmsSymbolFace{Z}}\xspace} % plain Z (no superscript 0)
\newcommand{\ttbar}{\ensuremath{{\cmsSymbolFace{t}\overline{\cmsSymbolFace{t}}}}\xspace} % t-tbar

% Extensions for missing names in PENNAMES % note no xspace, to match syntax in PENNAMES
\newcommand{\cPgn}{\ensuremath{\nu}} % generic neutrino
\providecommand{\Pgn}{\ensuremath{\nu}} % generic neutrino
\newcommand{\cPagn}{\ensuremath{\overline{\nu}}} % generic neutrino
\providecommand{\Pagn}{\ensuremath{\overline{\nu}}} % generic anti-neutrino
\newcommand{\cPgg}{\ensuremath{\gamma}} % gamma
\newcommand{\cPJgy}{\ensuremath{\cmsSymbolFace{J}\hspace{-.08em}/\hspace{-.14em}\psi}} % J/Psi (no mass)
\newcommand{\cPZ}{\ensuremath{\cmsSymbolFace{Z}}} % plain Z (no superscript 0)
\newcommand{\cPZpr}{\ensuremath{\cmsSymbolFace{Z}'}} % plain Z'
\newcommand{\cPqt}{\ensuremath{\cmsSymbolFace{t}}} % t for t quark
\newcommand{\cPqb}{\ensuremath{\cmsSymbolFace{b}}} % b for b quark
\newcommand{\cPqc}{\ensuremath{\cmsSymbolFace{c}}} % c for c quark
\newcommand{\cPqs}{\ensuremath{\cmsSymbolFace{s}}} % s for s quark
\newcommand{\cPqu}{\ensuremath{\cmsSymbolFace{u}}} % u for u quark
\newcommand{\cPqd}{\ensuremath{\cmsSymbolFace{d}}} % d for d quark
\newcommand{\cPq}{\ensuremath{\cmsSymbolFace{q}}} % generic quark
\newcommand{\cPg}{\ensuremath{\cmsSymbolFace{g}}} % generic gluon
\newcommand{\cPG}{\ensuremath{\cmsSymbolFace{G}}} % Graviton
\newcommand{\cPaqt}{\ensuremath{\overline{\cmsSymbolFace{t}}}} % t for t anti-quark
\newcommand{\cPaqb}{\ensuremath{\overline{\cmsSymbolFace{b}}}} % b for b anti-quark
\newcommand{\cPaqc}{\ensuremath{\overline{\cmsSymbolFace{c}}}} % c for c anti-quark
\newcommand{\cPaqs}{\ensuremath{\overline{\cmsSymbolFace{s}}}} % s for s anti-quark
\newcommand{\cPaqu}{\ensuremath{\overline{\cmsSymbolFace{u}}}} % u for u anti-quark
\newcommand{\cPaqd}{\ensuremath{\overline{\cmsSymbolFace{d}}}} % d for d anti-quark
\newcommand{\cPaq}{\ensuremath{\overline{\cmsSymbolFace{q}}}} % generic anti-quark
\newcommand{\cPKstz}{\ensuremath{\cmsSymbolFace{K}^{\ast0}}\xspace} %note has xspace
% future symbols from heppennames2
\providecommand{\PGp}{\ensuremath{\pi}\xspace} % pi
\providecommand{\PGpp}{\ensuremath{\pi^+}\xspace} % pi
\providecommand{\PGpm}{\ensuremath{\pi^-}\xspace} % pi
\providecommand{\PGpz}{\ensuremath{\pi^0}\xspace} % pi
\providecommand{\PGr}{\ensuremath{\rho}\xspace} % pi
\providecommand{\PDast}{\ensuremath{\cmsSymbolFace{D}^\ast}\xspace} % D star
\providecommand{\PH}{\ensuremath{\cmsSymbolFace{H}}\xspace} % plain Higgs
\providecommand{\Ph}{\ensuremath{\cmsSymbolFace{h}}\xspace} % SUSY Higgs
\providecommand{\Pa}{\ensuremath{\cmsSymbolFace{a}}\xspace}
\providecommand{\PSA}{\ensuremath{\cmsSymbolFace{A}}\xspace} %pseudoscalar A Higgs
\providecommand{\PJGy}{\ensuremath{\cmsSymbolFace{J}\hspace{-.08em}/\hspace{-.14em}\psi}\xspace} % J/Psi (no mass)
\providecommand{\PBzs}{\ensuremath{\cmsSymbolFace{B}^0_\cmsSymbolFace{s}}\xspace} % B^0_s
\providecommand{\Pg}{\ensuremath{\cmsSymbolFace{g}}\xspace} % generic gluon
\providecommand{\PSg}{\ensuremath{\widetilde{\cmsSymbolFace{g}}}\xspace} % gluino
\providecommand{\PSQ}{\ensuremath{\widetilde{\cmsSymbolFace{q}}}\xspace} % squark
\providecommand{\PSGm}{\ensuremath{\widetilde{\mu}}\xspace} % smuon
\providecommand{\PSe}{\ensuremath{\widetilde{\cmsSymbolFace{e}}}\xspace} % selectron
\providecommand{\PASQ}{\ensuremath{\overline{\widetilde{\cmsSymbolFace{q}}}}\xspace} % anti quark
\providecommand{\PXXA}{\ensuremath{\cmsSymbolFace{A}}\xspace} % axion
\providecommand{\PXXG}{\ensuremath{\cmsSymbolFace{G}}\xspace} % graviton
\providecommand{\PXXSG}{\ensuremath{\widetilde{\PXXG}}\xspace} % gravitino
\providecommand{\PSGcp}{\ensuremath{\widetilde{\chi}^+}\xspace} % lightest positive chargino
\providecommand{\PSGcm}{\ensuremath{\widetilde{\chi}^-}\xspace} % lightest negative chargino
\providecommand{\PSGc}{\ensuremath{\widetilde{\chi}}\xspace} % neutralino
\providecommand{\PSGcz}{\ensuremath{\widetilde{\chi}^0}\xspace} % neutralino with superscript 0
\providecommand{\PSGczDo}{\ensuremath{\widetilde{\chi}^{0}_{1}}\xspace} % neutralino
\providecommand{\PSGcmDo}{\ensuremath{\widetilde{\chi}^{-}_{1}}\xspace} % neutralino
\providecommand{\PSGczDt}{\ensuremath{\widetilde{\chi}^{0}_{2}}\xspace} % neutralino
\providecommand{\PSGcpm}{\ensuremath{\widetilde{\chi}^\pm}\xspace} % neutralino
\providecommand{\PSGcpmDo}{\ensuremath{\widetilde{\chi}^\pm_{1}}\xspace} % neutralino
\providecommand{\PSGcpDo}{\ensuremath{\widetilde{\chi}^{+}_{1}}\xspace} % neutralino
\providecommand{\Pl}{\ensuremath{\cmsSymbolFace{l}}\xspace} % non-ell lepton
\providecommand{\PAl}{\ensuremath{\overline{\cmsSymbolFace{l}}}\xspace} % non-ell anti-lepton
\providecommand{\PGnl}{\ensuremath{\nu_\cmsSymbolFace{l}}\xspace} % lepton neutrino
\providecommand{\PAGnl}{\ensuremath{\overline{\nu}_\cmsSymbolFace{l}}\xspace} % anti-lepton neutrino
\providecommand{\PQtpr}{\ensuremath{\cmsSymbolFace{t}^{\prime}}\xspace} % t'
\providecommand{\PAQtpr}{\ensuremath{\bar{\cmsSymbolFace{t}}^\prime}\xspace} % t'-bar; needs to be converted to overline-requires rework a la heppennames
\providecommand{\PQbpr}{\ensuremath{\cmsSymbolFace{b}^{\prime}}\xspace} % b'
\providecommand{\PAQbpr}{\ensuremath{\bar{\cmsSymbolFace{b}}^\prime}\xspace} % b'-bar; needs same as anti-t'
\providecommand{\PGg}{\ensuremath{\gamma}\xspace} % gamma
\providecommand{\PKzS}{\ensuremath{\cmsSymbolFace{K}^0_\cmsSymbolFace{S}}\xspace} % K short
\providecommand{\PBs}{\ensuremath{\cmsSymbolFace{B}_\cmsSymbolFace{s}}\xspace} % B sub s
\providecommand{\PSQu}{\ensuremath{\widetilde{\cmsSymbolFace{u}}}\xspace}
\providecommand{\PSQd}{\ensuremath{\widetilde{\cmsSymbolFace{d}}}\xspace}
\providecommand{\PSQc}{\ensuremath{\widetilde{\cmsSymbolFace{c}}}\xspace}
\providecommand{\PSQs}{\ensuremath{\widetilde{\cmsSymbolFace{s}}}\xspace}
\providecommand{\PSQt}{\ensuremath{\widetilde{\cmsSymbolFace{t}}}\xspace} % stop
\providecommand{\PSQb}{\ensuremath{\widetilde{\cmsSymbolFace{b}}}\xspace}
\providecommand{\PASQt}{\ensuremath{\overline{\widetilde{\cmsSymbolFace{t}}}}\xspace} % anti stop
\providecommand{\PASQb}{\ensuremath{\overline{\widetilde{\cmsSymbolFace{b}}}}\xspace} % anti sbottom
\providecommand{\PSGt}{\ensuremath{\widetilde{\tau}}\xspace} % stau
\providecommand{\PZ}{\ensuremath{\cmsSymbolFace{Z}}\xspace} % may have some confusion with the \xspace...
\providecommand{\PZpr}{\ensuremath{\cmsSymbolFace{Z}'}\xspace} % plain Z' using prime
% \renewcommand{\PWpr}{\ensuremath{\cmsSymbolFace{W}'}\xspace} % use prime like pennames2
\providecommand{\PWmp}{\ensuremath{\cmsSymbolFace{W}^\mp}\xspace}
\providecommand{\PWpm}{\ensuremath{\cmsSymbolFace{W}^\pm}\xspace}
\providecommand{\PDstp}{\ensuremath{\cmsSymbolFace{D}^{\ast+}}\xspace}
\providecommand{\PDstm}{\ensuremath{\cmsSymbolFace{D}^{\ast-}}\xspace}
\providecommand{\PGn}{\ensuremath{\nu}\xspace} % generic neutrino
\providecommand{\PAGn}{\ensuremath{\overline{\nu}}\xspace} % generic neutrino
\providecommand{\PSQtDo}{\ensuremath{\widetilde{\cmsSymbolFace{t}}_1}\xspace}
\providecommand{\PSQtDt}{\ensuremath{\widetilde{\cmsSymbolFace{t}}_2}\xspace}
\providecommand{\PQt}{\ensuremath{\cmsSymbolFace{t}}\xspace} % t
\providecommand{\PAQt}{\ensuremath{\overline{\cmsSymbolFace{t}}}\xspace} %
\providecommand{\PQb}{\ensuremath{\cmsSymbolFace{b}}\xspace} % b
\providecommand{\PAQb}{\ensuremath{\overline{\cmsSymbolFace{b}}}\xspace} %
\providecommand{\PGm}{\ensuremath{\mu}\xspace} % muon
\providecommand{\PGmm}{\ensuremath{\mu^-}\xspace} % muon
\providecommand{\PGmp}{\ensuremath{\mu^+}\xspace} % muon
\providecommand{\PGmpm}{\ensuremath{\mu^\pm}\xspace} % muon
\providecommand{\PGt}{\ensuremath{\tau}\xspace} % tau
\providecommand{\PAGt}{\ensuremath{\overline{\tau}}\xspace} % anti-tau
\providecommand{\PGpz}{\ensuremath{\pi^0}\xspace}
\providecommand{\PQq}{\ensuremath{\cmsSymbolFace{q}}\xspace} % quark (generic)
\providecommand{\PQd}{\ensuremath{\cmsSymbolFace{d}}\xspace} % down quark
\providecommand{\PQu}{\ensuremath{\cmsSymbolFace{u}}\xspace} % up quark
\providecommand{\PQs}{\ensuremath{\cmsSymbolFace{s}}\xspace} % top quark
\providecommand{\PQc}{\ensuremath{\cmsSymbolFace{c}}\xspace} % top quark
\providecommand{\PAQq}{\ensuremath{\overline{\cmsSymbolFace{q}}}\xspace} % quark (generic)
\providecommand{\PAQd}{\ensuremath{\overline{\cmsSymbolFace{d}}}\xspace} % down quark
\providecommand{\PAQu}{\ensuremath{\overline{\cmsSymbolFace{u}}}\xspace} % up quark
\providecommand{\PAQs}{\ensuremath{\overline{\cmsSymbolFace{s}}}\xspace} % top quark
\providecommand{\PAQc}{\ensuremath{\overline{\cmsSymbolFace{c}}}\xspace} % top quark
\providecommand{\PGne}{\ensuremath{\nu_\cmsSymbolFace{e}}\xspace} % electron neutrino
\providecommand{\PAGne}{\ensuremath{\overline{\nu}_\cmsSymbolFace{e}}\xspace} % anti-electron neutrino
\providecommand{\PGnGm}{\ensuremath{\nu_\PGm}\xspace} % muon neutrino
\providecommand{\PAGnGm}{\ensuremath{\overline{\nu}_\PGm}\xspace} % anti-muon neutrino
\providecommand{\PGnGt}{\ensuremath{\nu_\PGt}\xspace} % tau neutrino
\providecommand{\PAGnGt}{\ensuremath{\overline{\nu}_\PGt}\xspace} % anti-tau neutrino
\providecommand{\PAp}{\ensuremath{\overline{\cmsSymbolFace{p}}}\xspace} % anti-proton
\providecommand{\PAn}{\ensuremath{\overline{\cmsSymbolFace{n}}}\xspace} % anti-neutron
\providecommand{\PGc}{\ensuremath{\chi}\xspace} % chi (charm, but also SUSY)
\providecommand{\PGcc}{\ensuremath{\chi_{\PQc}}\xspace}
\providecommand{\PGcb}{\ensuremath{\chi_{\PQb}}\xspace}
\providecommand{\PDz}{\ensuremath{\cmsSymbolFace{D}^0}\xspace} % D0 meson
\providecommand{\PADz}{\ensuremath{\overline{\cmsSymbolFace{D}}^0}\xspace} % anti-D0 meson
\providecommand{\PAD}{\ensuremath{\overline{\cmsSymbolFace{D}}}\xspace} % anti-D meson
\providecommand{\PAK}{\ensuremath{\overline{\cmsSymbolFace{K}}}\xspace} % anti-K meson
\providecommand{\PAKz}{\ensuremath{\overline{\cmsSymbolFace{K}}^0}\xspace} % anti-K0 meson
\providecommand{\PABz}{\ensuremath{\overline{\cmsSymbolFace{B}}^0}\xspace} % anti-B0 meson
\providecommand{\PGLb}{\ensuremath{\Lambda_\cmsSymbolFace{b}}\xspace} % Lambda b

% our extensions for pennames2
\providecommand{\Pepm}{\ensuremath{\cmsSymbolFace{e}^\pm}\xspace}
\providecommand{\Pemp}{\ensuremath{\cmsSymbolFace{e}^\mp}\xspace}
\providecommand{\PGmpm}{\ensuremath{\mu^\pm}\xspace}
\providecommand{\PGmmp}{\ensuremath{\mu^\mp}\xspace} % not available in pennames2, AFAIK
% for APS style tables
\ifthenelse{\boolean{cms@external}}{%
 \newenvironment{scotch}[1]{\protect\centering\ruledtabular\tabular{#1}}{\endtabular\endruledtabular}
}{
 \newenvironment{scotch}[1]{\protect\centering\tabular{#1}\hline\hline}{\hline\endtabular}
}
% Other
\newcommand{\MD}{\ensuremath{{M_\mathrm{D}}}\xspace}% ED mass
\newcommand{\Mpl}{\ensuremath{{M_\mathrm{Pl}}}\xspace}% Planck mass
\newcommand{\Rinv} {\ensuremath{{R}^{-1}}\xspace}

% SM (still to be classified)

\newcommand{\AFB}{\ensuremath{A_\text{FB}}\xspace}
\newcommand{\wangle}{\ensuremath{\sin^{2}\theta_{\text{eff}}^\text{lept}(M^2_{\Z})}\xspace}
\newcommand{\stat}{\ensuremath{\,\text{(stat)}}\xspace}
\newcommand{\syst}{\ensuremath{\,\text{(syst)}}\xspace}
\newcommand{\thy}{\ensuremath{\,\text{(theo)}}\xspace}
\newcommand{\lum}{\ensuremath{\,\text{(lumi)}}\xspace}
\newcommand{\kt}{\ensuremath{k_{\mathrm{T}}}\xspace}

\newcommand{\BC}{\ensuremath{\cmsSymbolFace{B_{c}}}\xspace}
\newcommand{\bbarc}{\ensuremath{\PQb\PAQc}\xspace}
\newcommand{\bbbar}{\ensuremath{\PQb\PAQb}\xspace}
\newcommand{\ccbar}{\ensuremath{\PQc\PAQc}\xspace}
\newcommand{\qqbar}{\ensuremath{\PQq\PAQq}\xspace}
\newcommand{\bspsiphi}{\ensuremath{\cmsSymbolFace{B_s} \to \JPsi\, \phi}\xspace}
\newcommand{\EE}{\ensuremath{\Pep\Pem}\xspace}
\newcommand{\MM}{\ensuremath{\Pgmp\Pgmm}\xspace}
\newcommand{\TT}{\ensuremath{\Pgt^{+}\Pgt^{-}}\xspace}

%%%  E-gamma definitions
\newcommand{\HGG}{\ensuremath{\cmsSymbolFace{H}\to\gamma\gamma}}
\newcommand{\GAMJET}{\ensuremath{\gamma + \text{jet}}}
\newcommand{\PPTOJETS}{\ensuremath{\Pp\Pp\to\text{jets}}}
\newcommand{\PPTOGG}{\ensuremath{\Pp\Pp\to\gamma\gamma}}
\newcommand{\PPTOGAMJET}{\ensuremath{\Pp\Pp\to\gamma + \text{jet}}}
\newcommand{\MH}{\ensuremath{M_{\PH}}}
\newcommand{\RNINE}{\ensuremath{R_\mathrm{9}}}
\newcommand{\DR}{\ensuremath{\Delta R}}





%%%%%%
% From Albert
%

\newcommand{\ga}{\ensuremath{\gtrsim}}
\newcommand{\la}{\ensuremath{\lesssim}}
%
\newcommand{\swsq}{\ensuremath{\sin^2\theta_\cmsSymbolFace{W}}\xspace}
\newcommand{\cwsq}{\ensuremath{\cos^2\theta_\cmsSymbolFace{W}}\xspace}
\newcommand{\tanb}{\ensuremath{\tan\beta}\xspace}
\newcommand{\tanbsq}{\ensuremath{\tan^{2}\beta}\xspace}
\newcommand{\sidb}{\ensuremath{\sin 2\beta}\xspace}
\newcommand{\alpS}{\ensuremath{\alpha_S}\xspace}
\newcommand{\alpt}{\ensuremath{\tilde{\alpha}}\xspace}

\newcommand{\QL}{\ensuremath{\cmsSymbolFace{Q}_\cmsSymbolFace{L}}\xspace}
\newcommand{\sQ}{\ensuremath{\widetilde{\cmsSymbolFace{Q}}}\xspace}
\newcommand{\sQL}{\ensuremath{\widetilde{\cmsSymbolFace{Q}}_\cmsSymbolFace{L}}\xspace}
\newcommand{\ULC}{\ensuremath{\cmsSymbolFace{U}_\cmsSymbolFace{L}^\cmsSymbolFace{C}}\xspace}
\newcommand{\sUC}{\ensuremath{\widetilde{\cmsSymbolFace{U}}^\cmsSymbolFace{C}}\xspace}
\newcommand{\sULC}{\ensuremath{\widetilde{\cmsSymbolFace{U}}_\cmsSymbolFace{L}^\cmsSymbolFace{C}}\xspace}
\newcommand{\DLC}{\ensuremath{\cmsSymbolFace{D}_\cmsSymbolFace{L}^\cmsSymbolFace{C}}\xspace}
\newcommand{\sDC}{\ensuremath{\widetilde{\cmsSymbolFace{D}}^\cmsSymbolFace{C}}\xspace}
\newcommand{\sDLC}{\ensuremath{\widetilde{\cmsSymbolFace{D}}_\cmsSymbolFace{L}^\cmsSymbolFace{C}}\xspace}
\newcommand{\LL}{\ensuremath{\cmsSymbolFace{L}_\cmsSymbolFace{L}}\xspace}
\newcommand{\sL}{\ensuremath{\widetilde{\cmsSymbolFace{L}}}\xspace}
\newcommand{\sLL}{\ensuremath{\widetilde{\cmsSymbolFace{L}}_\cmsSymbolFace{L}}\xspace}
\newcommand{\ELC}{\ensuremath{\cmsSymbolFace{E}_\cmsSymbolFace{L}^\cmsSymbolFace{C}}\xspace}
\newcommand{\sEC}{\ensuremath{\widetilde{\cmsSymbolFace{E}}^\cmsSymbolFace{C}}\xspace}
\newcommand{\sELC}{\ensuremath{\widetilde{\cmsSymbolFace{E}}_\cmsSymbolFace{L}^\cmsSymbolFace{C}}\xspace}
\newcommand{\sEL}{\ensuremath{\widetilde{\cmsSymbolFace{E}}_\cmsSymbolFace{L}}\xspace}
\newcommand{\sER}{\ensuremath{\widetilde{\cmsSymbolFace{E}}_\cmsSymbolFace{R}}\xspace}
\newcommand{\sFer}{\ensuremath{\widetilde{\cmsSymbolFace{f}}}\xspace}
\newcommand{\sQua}{\ensuremath{\widetilde{\cmsSymbolFace{q}}}\xspace}
\newcommand{\sUp}{\ensuremath{\widetilde{\cmsSymbolFace{u}}}\xspace}
\newcommand{\suL}{\ensuremath{\widetilde{\cmsSymbolFace{u}}_\cmsSymbolFace{L}}\xspace}
\newcommand{\suR}{\ensuremath{\widetilde{\cmsSymbolFace{u}}_\cmsSymbolFace{R}}\xspace}
\newcommand{\sDw}{\ensuremath{\widetilde{\cmsSymbolFace{d}}}\xspace}
\newcommand{\sdL}{\ensuremath{\widetilde{\cmsSymbolFace{d}}_\cmsSymbolFace{L}}\xspace}
\newcommand{\sdR}{\ensuremath{\widetilde{\cmsSymbolFace{d}}_\cmsSymbolFace{R}}\xspace}
\newcommand{\sTop}{\ensuremath{\widetilde{\cmsSymbolFace{t}}}\xspace}
\newcommand{\stL}{\ensuremath{\widetilde{\cmsSymbolFace{t}}_\cmsSymbolFace{L}}\xspace}
\newcommand{\stR}{\ensuremath{\widetilde{\cmsSymbolFace{t}}_\cmsSymbolFace{R}}\xspace}
\newcommand{\stone}{\ensuremath{\widetilde{\cmsSymbolFace{t}}_1}\xspace}
\newcommand{\sttwo}{\ensuremath{\widetilde{\cmsSymbolFace{t}}_2}\xspace}
\newcommand{\sBot}{\ensuremath{\widetilde{\cmsSymbolFace{b}}}\xspace}
\newcommand{\sbL}{\ensuremath{\widetilde{\cmsSymbolFace{b}}_\cmsSymbolFace{L}}\xspace}
\newcommand{\sbR}{\ensuremath{\widetilde{\cmsSymbolFace{b}}_\cmsSymbolFace{R}}\xspace}
\newcommand{\sbone}{\ensuremath{\widetilde{\cmsSymbolFace{b}}_1}\xspace}
\newcommand{\sbtwo}{\ensuremath{\widetilde{\cmsSymbolFace{b}}_2}\xspace}
\newcommand{\sLep}{\ensuremath{\widetilde{\cmsSymbolFace{l}}}\xspace}
\newcommand{\sLepC}{\ensuremath{\widetilde{\cmsSymbolFace{l}}^\cmsSymbolFace{C}}\xspace}
\newcommand{\sEl}{\ensuremath{\widetilde{\cmsSymbolFace{e}}}\xspace}
\newcommand{\sElC}{\ensuremath{\widetilde{\cmsSymbolFace{e}}^\cmsSymbolFace{C}}\xspace}
\newcommand{\seL}{\ensuremath{\widetilde{\cmsSymbolFace{e}}_\cmsSymbolFace{L}}\xspace}
\newcommand{\seR}{\ensuremath{\widetilde{\cmsSymbolFace{e}}_\cmsSymbolFace{R}}\xspace}
\newcommand{\snL}{\ensuremath{\widetilde{\nu}_L}\xspace}
\newcommand{\sMu}{\ensuremath{\widetilde{\mu}}\xspace}
\newcommand{\sNu}{\ensuremath{\widetilde{\nu}}\xspace}
\newcommand{\sTau}{\ensuremath{\widetilde{\tau}}\xspace}
\newcommand{\Glu}{\ensuremath{\cmsSymbolFace{g}}\xspace}
\newcommand{\sGlu}{\ensuremath{\widetilde{\cmsSymbolFace{g}}}\xspace}
\newcommand{\Wpm}{\ensuremath{\cmsSymbolFace{W}^{\pm}}\xspace}
\newcommand{\sWpm}{\ensuremath{\widetilde{\cmsSymbolFace{W}}^{\pm}}\xspace}
\newcommand{\Wz}{\ensuremath{\cmsSymbolFace{W}^{0}}\xspace}
\newcommand{\sWz}{\ensuremath{\widetilde{\cmsSymbolFace{W}}^{0}}\xspace}
\newcommand{\sWino}{\ensuremath{\widetilde{\cmsSymbolFace{W}}}\xspace}
\newcommand{\Bz}{\ensuremath{\cmsSymbolFace{B}^{0}}\xspace}
\newcommand{\sBz}{\ensuremath{\widetilde{\cmsSymbolFace{B}}^{0}}\xspace}
\newcommand{\sBino}{\ensuremath{\widetilde{\cmsSymbolFace{B}}}\xspace}
\newcommand{\Zz}{\ensuremath{\cmsSymbolFace{Z}^{0}}\xspace}
\newcommand{\sZino}{\ensuremath{\widetilde{\cmsSymbolFace{Z}}^{0}}\xspace}
\newcommand{\sGam}{\ensuremath{\widetilde{\gamma}}\xspace}
\newcommand{\chiz}{\ensuremath{\widetilde{\chi}^{0}}\xspace}
\newcommand{\chip}{\ensuremath{\widetilde{\chi}^{+}}\xspace}
\newcommand{\chim}{\ensuremath{\widetilde{\chi}^{-}}\xspace}
\newcommand{\chipm}{\ensuremath{\widetilde{\chi}^{\pm}}\xspace}
\newcommand{\Hone}{\ensuremath{\cmsSymbolFace{H}_\cmsSymbolFace{d}}\xspace}
\newcommand{\sHone}{\ensuremath{\widetilde{\cmsSymbolFace{H}}_\cmsSymbolFace{d}}\xspace}
\newcommand{\Htwo}{\ensuremath{\cmsSymbolFace{H}_\cmsSymbolFace{u}}\xspace}
\newcommand{\sHtwo}{\ensuremath{\widetilde{\cmsSymbolFace{H}}_\cmsSymbolFace{u}}\xspace}
\newcommand{\sHig}{\ensuremath{\widetilde{\cmsSymbolFace{H}}}\xspace}
\newcommand{\sHa}{\ensuremath{\widetilde{\cmsSymbolFace{H}}_\cmsSymbolFace{a}}\xspace}
\newcommand{\sHb}{\ensuremath{\widetilde{\cmsSymbolFace{H}}_\cmsSymbolFace{b}}\xspace}
\newcommand{\sHpm}{\ensuremath{\widetilde{\cmsSymbolFace{H}}^{\pm}}\xspace}
\newcommand{\hz}{\ensuremath{\cmsSymbolFace{h}^{0}}\xspace}
\newcommand{\Hz}{\ensuremath{\cmsSymbolFace{H}^{0}}\xspace}
\newcommand{\Az}{\ensuremath{\cmsSymbolFace{A}^{0}}\xspace}
\newcommand{\Hpm}{\ensuremath{\cmsSymbolFace{H}^{\pm}}\xspace}
\newcommand{\sGra}{\ensuremath{\widetilde{\cmsSymbolFace{G}}}\xspace}
%
\newcommand{\mtil}{\ensuremath{\widetilde{m}}\xspace}
%
\newcommand{\rpv}{\ensuremath{\rlap{\kern.2em/}R}\xspace}
\newcommand{\LLE}{\ensuremath{LL\bar{E}}\xspace}
\newcommand{\LQD}{\ensuremath{LQ\bar{D}}\xspace}
\newcommand{\UDD}{\ensuremath{\overline{UDD}}\xspace}
\newcommand{\Lam}{\ensuremath{\lambda}\xspace}
\newcommand{\Lamp}{\ensuremath{\lambda'}\xspace}
\newcommand{\Lampp}{\ensuremath{\lambda''}\xspace}
%
\newcommand{\spinbd}[2]{\ensuremath{\bar{#1}_{\dot{#2}}}\xspace}

\endinput
 	% official CMS definitions
% units and symbols
% most of them are defined in ptdr-definitions.tex, the ones that are missing there are defined here
\newcommand{\pb}{\ensuremath{\,\text{pb}}\xspace}
\newcommand{\MV}{\ensuremath{\,\text{MV}}\xspace}
\newcommand{\MVm}{\ensuremath{\,\text{MV\hspace{-0.16em}/\hspace{-0.08em}m}}\xspace}
\newcommand{\MHz}{\ensuremath{\,\text{MHz}}\xspace}
\newcommand{\T}{\ensuremath{\,\text{T}}\xspace}
\newcommand{\mrad}{\ensuremath{\,\text{mrad}}\xspace}
\newcommand{\mt}{\ensuremath{m_{\mathrm{T}}}\xspace}
\newcommand{\metSlash}{\ensuremath{{\not\mathrel{E}}_\mathrm{T}}} % alternative version to \ETslash, w/o spacing problem

% period and comma after formulas with some extra spacing
\newcommand{\paf}{\ .}
\newcommand{\caf}{\ ,}
\newcommand{\waf}[1]{\ \text{#1}}

% references
\AtBeginDocument{ % hyperref redefines \ref at beginning of the document
	\let\oldref\ref
	%%% PubCom recommendations are given as comments
	%%% Currently, I do not follow PubCom recommendations
	\newcommand{\refChap}[1]{\hyperref[#1]{Chapter~\oldref*{#1}}}		% 'Chapter 3'
	\newcommand{\refSec} [1]{\hyperref[#1]{Section~\oldref*{#1}}}		% 'Section 3'
	\newcommand{\refApp} [1]{\hyperref[#1]{Appendix~\oldref*{#1}}}		% 'Appendix B'
	\newcommand{\refFig} [1]{\hyperref[#1]{Figure~\oldref*{#1}}}		% 'Fig. 3'; at beginning of sentence 'Figure 3'
	\newcommand{\refTab} [1]{\hyperref[#1]{Table~\oldref*{#1}}}			% 'Table 3'
	\newcommand{\refEq}  [1]{\hyperref[#1]{Equation~(\oldref*{#1})}}% 'Eq. (3)'; at beginning of sentence 'Equation (3)'
}

%% singlets and doublets (for SM table)
\newcommand{\doublet}[2]{$\begin{pmatrix} #1 \\ #2 \end{pmatrix}_{\mathrm{L}}$}
\newcommand{\lsinglet}[1]{$\begin{array}{c} #1_\mathrm{R}^\mathrm{-}\end{array}$}
\newcommand{\qsinglet}[2]{$\begin{array}{c} #1_\mathrm{R} \\ #2_\mathrm{R}\end{array}$}
\newcommand{\doubarrc}[2]{$\begin{array}{c} #1 \\ #2  \end{array}$}
\newcommand{\doubarrl}[2]{$\hspace{-2ex}\begin{array}{l} #1 \\ #2  \end{array}$}
\newcommand{\doubarrr}[2]{$\begin{array}{r} #1 \\ #2  \end{array}\hspace{-2ex}$}
\newcommand{\singarrl}[1]{$\hspace{-2ex}\begin{array}{l} #1  \end{array}$}
\newcommand{\singarrr}[1]{$\begin{array}{r} #1  \end{array}\hspace{-2ex}$}

% defined to be equal symbol :=
\newcommand*{\defeq}{\mathrel{\vcenter{\baselineskip0.5ex \lineskiplimit0pt
                     \hbox{\scriptsize.}\hbox{\scriptsize.}}}%
                     =}
										
% ########## Hyphenations ##########
% \hyphenation{ex-am-ple}
		% personal definitions

%%%%%%%%%%%%%%%  Title page %%%%%%%%%%%%%%%%%%%%%%%%
%\documentclass[11pt,twoside, openright, a4paper, pdftex, tdr]{new-cms-tdr}
\documentclass[11pt,twoside, openright, a4paper, pdftex, tdr,chapterprefix=false]{new-cms-tdr}
%\usepackage[left=29mm,right=25mm,top=25mm,bottom=14mm,includeheadfoot]{geometry}%includehead
% \usepackage[left=3cm,right=3cm,top=3cm,bottom=3cm,includeheadfoot]{geometry}%includehead
\usepackage[left=3cm,right=2cm,top=3cm,bottom=3cm,includeheadfoot]{geometry}%includehead

%\usepackage{geometry} % see geometry.pdf on how to lay out the page. There's lots.
%\geometry{a4paper} % or letter or a5paper or ... etc
% \geometry{landscape} % rotated page geometry

\usepackage[latin1]{inputenc}

\usepackage{general/ptdr-definitions} 	% official CMS definitions


%%MEINE INCLUDES
\usepackage{subfigure}
\usepackage{amsmath}
\usepackage{graphics}
\usepackage{graphicx}
%\usepackage{caption}
%\usepackage{subcaption}
\usepackage{lmodern}
% \usepackage[ngerman]{babel}
\usepackage[english]{babel}
\usepackage{amssymb}
%\usepackage{hyperref}
%\usepackage{amssymb,amsmath,hyperref}
%\usepackage{svg}

\usepackage{siunitx}
%\newcommand{\pT}{\ensuremath{p_\perp}}
%\newcommand{\pT}{\ensuremath{p_T}}
%\newcommand{\Et}{\ensuremath{E_\perp}}
%\newcommand{\dpt}{\ensuremath{\delta \pT/\pT}}
%\newcommand{\dE}{\ensuremath{\delta E/E}}
%\newcommand{\nm}{\nano\meter}
%\newcommand{\kg}{\kilo\gram}
%\newcommand{\fb}{\femto\barn}
%\newcommand{\fbi}{\ensuremath{\fb^{-1}}}
%
%\newcommand{\pbi}{\ensuremath{\pb^{-1}}}
%\newcommand{\nb}{\nano\barn}
%\newcommand{\nbi}{\ensuremath{\nb^{-1}}}
%\def\Fileversion$#1: #2 ${\gdef\fileversion{#2}}
\def\Filedate$#1: #2-#3-#4 #5 ${\gdef\filedate{#2/#3/#4}}
\Fileversion$Revision: 466930 $
\Filedate$Date: 2018-06-30 19:31:24 +0200 (Sa, 30 Jun 2018) $
%%%%%%%%%%%%%%%%%%%%%%%%%%%%%%%%%%%%%%%%%%%%%%%%%%%%%%%%%%%%%%%%%%%%
%
%  CMS Common definitions style file
%
%  N.B. use of \newcommand rather than \newcommand means
%       that a definition is ignored if already specified
%
%                                              L. Taylor 18 Feb 2005
%%%%%%%%%%%%%%%%%%%%%%%%%%%%%%%%%%%%%%%%%%%%%%%%%%%%%%%%%%%%%%%%%%%%
\NeedsTeXFormat{LaTeX2e}
\ProvidesPackage{ptdr-definitions}[\filedate\space CMS Additional Macro Definitions (\fileversion)]
\RequirePackage{xspace}
\RequirePackage{amsmath}

% Some shorthand
% turn off italics
\newcommand {\etal}{\mbox{et al.}\xspace} %et al. - no preceding comma
\newcommand {\ie}{\mbox{i.e.}\xspace}     %i.e.
\newcommand {\eg}{\mbox{e.g.}\xspace}     %e.g.
\newcommand {\etc}{\mbox{etc.}\xspace}     %etc.
\newcommand {\vs}{\mbox{\sl vs.}\xspace}      %vs.
\newcommand {\mdash}{\ensuremath{\mathrm{-}}} % for use within formulas
\providecommand {\NA}{\ensuremath{\text{---}}}    % for Not applicable (or available). Needs to be renewcommanded for APS to \cdots

% some terms whose definition we may change
\newcommand {\Lone}{Level-1\xspace} % Level-1 or L1 ?
\newcommand {\Ltwo}{Level-2\xspace}
\newcommand {\Lthree}{Level-3\xspace}

% Some software programs (alphabetized)
\newcommand{\ACERMC} {\textsc{AcerMC}\xspace}
\newcommand{\ALPGEN} {{\textsc{alpgen}}\xspace}
\newcommand{\BLACKHAT} {{\textsc{BlackHat}}\xspace}
\newcommand{\CALCHEP} {{\textsc{CalcHEP}}\xspace}
\newcommand{\CHARYBDIS} {{\textsc{charybdis}}\xspace}
\newcommand{\CMKIN} {\textsc{cmkin}\xspace}
\newcommand{\CMSIM} {{\textsc{cmsim}}\xspace}
\newcommand{\CMSSW} {{\textsc{cmssw}}\xspace}
\newcommand{\COBRA} {{\textsc{cobra}}\xspace}
\newcommand{\COCOA} {{\textsc{cocoa}}\xspace}
\newcommand{\COMPHEP} {\textsc{CompHEP}\xspace}
\newcommand{\EVTGEN} {{\textsc{evtgen}}\xspace}
\newcommand{\FAMOS} {{\textsc{famos}}\xspace}
\newcommand{\FASTJET} {{\textsc{FastJet}}\xspace}
\newcommand{\FEWZ} {{\textsc{fewz}}\xspace}
\newcommand{\GARCON} {\textsc{garcon}\xspace}
\newcommand{\GARFIELD} {{\textsc{garfield}}\xspace}
\newcommand{\GEANE} {{\textsc{geane}}\xspace}
\newcommand{\GEANTfour} {{\textsc{Geant4}}\xspace}
\newcommand{\GEANTthree} {{\textsc{geant3}}\xspace}
\newcommand{\GEANT} {{\textsc{geant}}\xspace}
\newcommand{\HDECAY} {\textsc{hdecay}\xspace}
\newcommand{\HERWIG} {{\textsc{herwig}}\xspace}
\newcommand{\HERWIGpp} {{\textsc{herwig++}}\xspace}
\newcommand{\POWHEG} {{\textsc{powheg}}\xspace}
\newcommand{\HIGLU} {{\textsc{higlu}}\xspace}
\newcommand{\HIJING} {{\textsc{hijing}}\xspace}
\newcommand{\HYDJET} {{\textsc{hydjet}}\xspace}
\newcommand{\IGUANA} {\textsc{iguana}\xspace}
\newcommand{\ISAJET} {{\textsc{isajet}}\xspace}
\newcommand{\ISAPYTHIA} {{\textsc{isapythia}}\xspace}
\newcommand{\ISASUGRA} {{\textsc{isasugra}}\xspace}
\newcommand{\ISASUSY} {{\textsc{isasusy}}\xspace}
\newcommand{\ISAWIG} {{\textsc{isawig}}\xspace}
\newcommand{\MADGRAPH} {\textsc{MadGraph}\xspace}
\newcommand{\MCATNLO} {\textsc{mc@nlo}\xspace}
\newcommand{\MCFM} {\textsc{mcfm}\xspace}
\newcommand{\MILLEPEDE} {{\textsc{millepede}}\xspace}
\newcommand{\ORCA} {{\textsc{orca}}\xspace}
\newcommand{\OSCAR} {{\textsc{oscar}}\xspace}
\newcommand{\PHOTOS} {\textsc{photos}\xspace}
\newcommand{\PROSPINO} {\textsc{prospino}\xspace}
\newcommand{\PYTHIA} {{\textsc{pythia}}\xspace}
\newcommand{\SHERPA} {{\textsc{sherpa}}\xspace}
\newcommand{\TAUOLA} {\textsc{tauola}\xspace}
\newcommand{\TOPREX} {\textsc{TopReX}\xspace}
\newcommand{\XDAQ} {{\textsc{xdaq}}\xspace}
\newcommand{\MGvATNLO}{\MADGRAPH{}5\_a\MCATNLO}


%  Experiments
\newcommand {\DZERO}{D0\xspace}     %etc.


% Measurements and units...

\newcommand{\de}{\ensuremath{^\circ}}
\newcommand{\ten}[1]{\ensuremath{\times \text{10}^\text{#1}}}
\newcommand{\unit}[1]{\ensuremath{\text{\,#1}}\xspace}
\newcommand{\mum}{\ensuremath{\,\mu\text{m}}\xspace}
\newcommand{\micron}{\ensuremath{\,\mu\text{m}}\xspace}
\newcommand{\cm}{\ensuremath{\,\text{cm}}\xspace}
\newcommand{\m}{\ensuremath{\,\text{m}}\xspace}
\newcommand{\mm}{\ensuremath{\,\text{mm}}\xspace}
\newcommand{\km}{\ensuremath{\,\text{km}}\xspace}
\newcommand{\mus}{\ensuremath{\,\mu\text{s}}\xspace}
\newcommand{\keV}{\ensuremath{\,\text{ke\hspace{-.08em}V}}\xspace}
\newcommand{\MeV}{\ensuremath{\,\text{Me\hspace{-.08em}V}}\xspace}
\newcommand{\MeVns}{\ensuremath{\text{Me\hspace{-.08em}V}}\xspace} % no leading thinspace
\newcommand{\GeV}{\ensuremath{\,\text{Ge\hspace{-.08em}V}}\xspace}
% \newcommand{\e}{\ensuremath{\,\text{e\xspace}
\newcommand{\GeVns}{\ensuremath{\text{Ge\hspace{-.08em}V}}\xspace} % no leading thinspace
\newcommand{\gev}{\GeV}
\newcommand{\TeV}{\ensuremath{\,\text{Te\hspace{-.08em}V}}\xspace}
\newcommand{\TeVns}{\ensuremath{\text{Te\hspace{-.08em}V}}\xspace} % no leading thinspace
\newcommand{\PeV}{\ensuremath{\,\text{Pe\hspace{-.08em}V}}\xspace}
\newcommand{\keVc}{\ensuremath{{\,\text{ke\hspace{-.08em}V\hspace{-0.16em}/\hspace{-0.08em}}c}}\xspace}
\newcommand{\MeVc}{\ensuremath{{\,\text{Me\hspace{-.08em}V\hspace{-0.16em}/\hspace{-0.08em}}c}}\xspace}
\newcommand{\GeVc}{\ensuremath{{\,\text{Ge\hspace{-.08em}V\hspace{-0.16em}/\hspace{-0.08em}}c}}\xspace}
\newcommand{\GeVcns}{\ensuremath{{\text{Ge\hspace{-.08em}V\hspace{-0.16em}/\hspace{-0.08em}}c}}\xspace} % no leading thinspace
\newcommand{\TeVc}{\ensuremath{{\,\text{Te\hspace{-.08em}V\hspace{-0.16em}/\hspace{-0.08em}}c}}\xspace}
\newcommand{\keVcc}{\ensuremath{{\,\text{ke\hspace{-.08em}V\hspace{-0.16em}/\hspace{-0.08em}}c^\text{2}}}\xspace}
\newcommand{\MeVcc}{\ensuremath{{\,\text{Me\hspace{-.08em}V\hspace{-0.16em}/\hspace{-0.08em}}c^\text{2}}}\xspace}
\newcommand{\GeVcc}{\ensuremath{{\,\text{Ge\hspace{-.08em}V\hspace{-0.16em}/\hspace{-0.08em}}c^\text{2}}}\xspace}
\newcommand{\GeVccns}{\ensuremath{{\text{Ge\hspace{-.08em}V\hspace{-0.16em}/\hspace{-0.08em}}c^\text{2}}}\xspace} % no leading thinspace
\newcommand{\TeVcc}{\ensuremath{{\,\text{Te\hspace{-.08em}V\hspace{-0.16em}/\hspace{-0.08em}}c^\text{2}}}\xspace}

\newcommand{\pbinv} {\mbox{\ensuremath{\,\text{pb}^\text{$-$1}}}\xspace}
\newcommand{\binv} {\mbox{\ensuremath{\,\text{b}^\text{$-$1}}}\xspace}
\newcommand{\fbinv} {\mbox{\ensuremath{\,\text{fb}^\text{$-$1}}}\xspace}
\newcommand{\nbinv} {\mbox{\ensuremath{\,\text{nb}^\text{$-$1}}}\xspace}
\newcommand{\mubinv} {\ensuremath{\,\mu\mathrm{b}^{-1}}\xspace}
\newcommand{\mbinv} {\ensuremath{\,\mathrm{mb}^{-1}}\xspace}
\newcommand{\percms}{\ensuremath{\,\text{cm}^\text{$-$2}\,\text{s}^\text{$-$1}}\xspace}
\newcommand{\lumi}{\ensuremath{\mathcal{L}}\xspace}
\newcommand{\Lumi}{\ensuremath{\mathcal{L}}\xspace}%both upper and lower
%
% Need a convention here:
\newcommand{\LvLow}  {\ensuremath{\mathcal{L}=\text{10}^\text{32}\,\text{cm}^\text{$-$2}\,\text{s}^\text{$-$1}}\xspace}
\newcommand{\LLow}   {\ensuremath{\mathcal{L}=\text{10}^\text{33}\,\text{cm}^\text{$-$2}\,\text{s}^\text{$-$1}}\xspace}
\newcommand{\lowlumi}{\ensuremath{\mathcal{L}=\text{2}\times \text{10}^\text{33}\,\text{cm}^\text{$-$2}\,\text{s}^\text{$-$1}}\xspace}
\newcommand{\LMed}   {\ensuremath{\mathcal{L}=\text{2}\times \text{10}^\text{33}\,\text{cm}^\text{$-$2}\,\text{s}^\text{$-$1}}\xspace}
\newcommand{\LHigh}  {\ensuremath{\mathcal{L}=\text{10}^\text{34}\,\text{cm}^\text{$-$2}\,\text{s}^\text{$-$1}}\xspace}
\newcommand{\hilumi} {\ensuremath{\mathcal{L}=\text{10}^\text{34}\,\text{cm}^\text{$-$2}\,\text{s}^\text{$-$1}}\xspace}

% Physics symbols ...

\newcommand{\PT}{\ensuremath{p_{\mathrm{T}}}\xspace}
\newcommand{\pt}{\ensuremath{p_{\mathrm{T}}}\xspace}
\newcommand{\ET}{\ensuremath{E_{\mathrm{T}}}\xspace}
\newcommand{\HT}{\ensuremath{H_{\mathrm{T}}}\xspace}
\newcommand{\mT}{\ensuremath{m_{\mathrm{T}}}\xspace}
\newcommand{\mTii}{\ensuremath{m_{\mathrm{T2}}}\xspace}
\newcommand{\et}{\ensuremath{E_{\mathrm{T}}}\xspace}
\newcommand{\Em}{\ensuremath{E\hspace{-0.6em}/}\xspace}
\newcommand{\Pm}{\ensuremath{p\hspace{-0.5em}/}\xspace}
\newcommand{\PTm}{\ensuremath{{p}_\mathrm{T}\hspace{-1.02em}/\kern 0.5em}\xspace}
\newcommand{\PTslash}{\PTm}
\newcommand{\ETm}{\ensuremath{E_{\mathrm{T}}^{\text{miss}}}\xspace}
\newcommand{\MET}{\ETm}
\newcommand{\ETmiss}{\ETm}
\newcommand{\ptmiss}{\ensuremath{\pt^\text{miss}}\xspace}
\newcommand{\ETslash}{\ensuremath{E_{\mathrm{T}}\hspace{-1.1em}/\kern0.45em}\xspace}
\newcommand{\VEtmiss}{\ensuremath{{\vec E}_{\mathrm{T}}^{\text{miss}}}\xspace}
\newcommand{\ptvec}{\ensuremath{{\vec p}_{\mathrm{T}}}\xspace}
\newcommand{\ptvecmiss}{\ensuremath{{\vec p}_{\mathrm{T}}^{\kern1pt\text{miss}}}\xspace}
\newcommand{\tauh}{\ensuremath{\PGt_\mathrm{h}}\xspace}
\newcommand{\sqrtsNN}{\ensuremath{\sqrt{\smash[b]{s_{_{\mathrm{NN}}}}}}\xspace}
\newcommand{\mht}{\ensuremath{H_{\mathrm{T}}^{\text{miss}}}\xspace}
\newcommand{\htvecmiss}{\ensuremath{\vec{H}_{\text{T}}^{\text{miss}}}\xspace}

% roman face derivative
\newcommand{\dd}[2]{\ensuremath{\frac{\cmsSymbolFace{d} #1}{\cmsSymbolFace{d} #2}}}
\newcommand{\ddinline}[2]{\ensuremath{\cmsSymbolFace{d} #1/\cmsSymbolFace{d} #2}}
\newcommand{\rd}{\ensuremath{\cmsSymbolFace{d}}}
\newcommand{\re}{\ensuremath{\cmsSymbolFace{e}}}
% absolute value
\newcommand{\abs}[1]{\ensuremath{\lvert #1 \rvert}}
% misc
\newcommand{\CL}{\ensuremath{\text{CL}}\xspace}
\newcommand{\CLs}{\ensuremath{\text{CL}_\text{s}}\xspace}
\newcommand{\CLsb}{\ensuremath{\text{CL}_\text{s+b}}\xspace}



\ifthenelse{\boolean{cms@italic}}{\newcommand{\cmsSymbolFace}[1]{#1}}{\newcommand{\cmsSymbolFace}[1]{\mathrm{#1}}}

% Particle names which track the italic/non-italic face convention
\newcommand{\zp}{\ensuremath{{\cmsSymbolFace{Z}^\prime}}\xspace} % plain Z'
\newcommand{\JPsi}{\ensuremath{{\cmsSymbolFace{J}\hspace{-.08em}/\hspace{-.14em}\psi}}\xspace} % J/Psi (no mass)
\newcommand{\Z}{\ensuremath{\cmsSymbolFace{Z}}\xspace} % plain Z (no superscript 0)
\newcommand{\ttbar}{\ensuremath{{\cmsSymbolFace{t}\overline{\cmsSymbolFace{t}}}}\xspace} % t-tbar

% Extensions for missing names in PENNAMES % note no xspace, to match syntax in PENNAMES
\newcommand{\cPgn}{\ensuremath{\nu}} % generic neutrino
\providecommand{\Pgn}{\ensuremath{\nu}} % generic neutrino
\newcommand{\cPagn}{\ensuremath{\overline{\nu}}} % generic neutrino
\providecommand{\Pagn}{\ensuremath{\overline{\nu}}} % generic anti-neutrino
\newcommand{\cPgg}{\ensuremath{\gamma}} % gamma
\newcommand{\cPJgy}{\ensuremath{\cmsSymbolFace{J}\hspace{-.08em}/\hspace{-.14em}\psi}} % J/Psi (no mass)
\newcommand{\cPZ}{\ensuremath{\cmsSymbolFace{Z}}} % plain Z (no superscript 0)
\newcommand{\cPZpr}{\ensuremath{\cmsSymbolFace{Z}'}} % plain Z'
\newcommand{\cPqt}{\ensuremath{\cmsSymbolFace{t}}} % t for t quark
\newcommand{\cPqb}{\ensuremath{\cmsSymbolFace{b}}} % b for b quark
\newcommand{\cPqc}{\ensuremath{\cmsSymbolFace{c}}} % c for c quark
\newcommand{\cPqs}{\ensuremath{\cmsSymbolFace{s}}} % s for s quark
\newcommand{\cPqu}{\ensuremath{\cmsSymbolFace{u}}} % u for u quark
\newcommand{\cPqd}{\ensuremath{\cmsSymbolFace{d}}} % d for d quark
\newcommand{\cPq}{\ensuremath{\cmsSymbolFace{q}}} % generic quark
\newcommand{\cPg}{\ensuremath{\cmsSymbolFace{g}}} % generic gluon
\newcommand{\cPG}{\ensuremath{\cmsSymbolFace{G}}} % Graviton
\newcommand{\cPaqt}{\ensuremath{\overline{\cmsSymbolFace{t}}}} % t for t anti-quark
\newcommand{\cPaqb}{\ensuremath{\overline{\cmsSymbolFace{b}}}} % b for b anti-quark
\newcommand{\cPaqc}{\ensuremath{\overline{\cmsSymbolFace{c}}}} % c for c anti-quark
\newcommand{\cPaqs}{\ensuremath{\overline{\cmsSymbolFace{s}}}} % s for s anti-quark
\newcommand{\cPaqu}{\ensuremath{\overline{\cmsSymbolFace{u}}}} % u for u anti-quark
\newcommand{\cPaqd}{\ensuremath{\overline{\cmsSymbolFace{d}}}} % d for d anti-quark
\newcommand{\cPaq}{\ensuremath{\overline{\cmsSymbolFace{q}}}} % generic anti-quark
\newcommand{\cPKstz}{\ensuremath{\cmsSymbolFace{K}^{\ast0}}\xspace} %note has xspace
% future symbols from heppennames2
\providecommand{\PGp}{\ensuremath{\pi}\xspace} % pi
\providecommand{\PGpp}{\ensuremath{\pi^+}\xspace} % pi
\providecommand{\PGpm}{\ensuremath{\pi^-}\xspace} % pi
\providecommand{\PGpz}{\ensuremath{\pi^0}\xspace} % pi
\providecommand{\PGr}{\ensuremath{\rho}\xspace} % pi
\providecommand{\PDast}{\ensuremath{\cmsSymbolFace{D}^\ast}\xspace} % D star
\providecommand{\PH}{\ensuremath{\cmsSymbolFace{H}}\xspace} % plain Higgs
\providecommand{\Ph}{\ensuremath{\cmsSymbolFace{h}}\xspace} % SUSY Higgs
\providecommand{\Pa}{\ensuremath{\cmsSymbolFace{a}}\xspace}
\providecommand{\PSA}{\ensuremath{\cmsSymbolFace{A}}\xspace} %pseudoscalar A Higgs
\providecommand{\PJGy}{\ensuremath{\cmsSymbolFace{J}\hspace{-.08em}/\hspace{-.14em}\psi}\xspace} % J/Psi (no mass)
\providecommand{\PBzs}{\ensuremath{\cmsSymbolFace{B}^0_\cmsSymbolFace{s}}\xspace} % B^0_s
\providecommand{\Pg}{\ensuremath{\cmsSymbolFace{g}}\xspace} % generic gluon
\providecommand{\PSg}{\ensuremath{\widetilde{\cmsSymbolFace{g}}}\xspace} % gluino
\providecommand{\PSQ}{\ensuremath{\widetilde{\cmsSymbolFace{q}}}\xspace} % squark
\providecommand{\PSGm}{\ensuremath{\widetilde{\mu}}\xspace} % smuon
\providecommand{\PSe}{\ensuremath{\widetilde{\cmsSymbolFace{e}}}\xspace} % selectron
\providecommand{\PASQ}{\ensuremath{\overline{\widetilde{\cmsSymbolFace{q}}}}\xspace} % anti quark
\providecommand{\PXXA}{\ensuremath{\cmsSymbolFace{A}}\xspace} % axion
\providecommand{\PXXG}{\ensuremath{\cmsSymbolFace{G}}\xspace} % graviton
\providecommand{\PXXSG}{\ensuremath{\widetilde{\PXXG}}\xspace} % gravitino
\providecommand{\PSGcp}{\ensuremath{\widetilde{\chi}^+}\xspace} % lightest positive chargino
\providecommand{\PSGcm}{\ensuremath{\widetilde{\chi}^-}\xspace} % lightest negative chargino
\providecommand{\PSGc}{\ensuremath{\widetilde{\chi}}\xspace} % neutralino
\providecommand{\PSGcz}{\ensuremath{\widetilde{\chi}^0}\xspace} % neutralino with superscript 0
\providecommand{\PSGczDo}{\ensuremath{\widetilde{\chi}^{0}_{1}}\xspace} % neutralino
\providecommand{\PSGcmDo}{\ensuremath{\widetilde{\chi}^{-}_{1}}\xspace} % neutralino
\providecommand{\PSGczDt}{\ensuremath{\widetilde{\chi}^{0}_{2}}\xspace} % neutralino
\providecommand{\PSGcpm}{\ensuremath{\widetilde{\chi}^\pm}\xspace} % neutralino
\providecommand{\PSGcpmDo}{\ensuremath{\widetilde{\chi}^\pm_{1}}\xspace} % neutralino
\providecommand{\PSGcpDo}{\ensuremath{\widetilde{\chi}^{+}_{1}}\xspace} % neutralino
\providecommand{\Pl}{\ensuremath{\cmsSymbolFace{l}}\xspace} % non-ell lepton
\providecommand{\PAl}{\ensuremath{\overline{\cmsSymbolFace{l}}}\xspace} % non-ell anti-lepton
\providecommand{\PGnl}{\ensuremath{\nu_\cmsSymbolFace{l}}\xspace} % lepton neutrino
\providecommand{\PAGnl}{\ensuremath{\overline{\nu}_\cmsSymbolFace{l}}\xspace} % anti-lepton neutrino
\providecommand{\PQtpr}{\ensuremath{\cmsSymbolFace{t}^{\prime}}\xspace} % t'
\providecommand{\PAQtpr}{\ensuremath{\bar{\cmsSymbolFace{t}}^\prime}\xspace} % t'-bar; needs to be converted to overline-requires rework a la heppennames
\providecommand{\PQbpr}{\ensuremath{\cmsSymbolFace{b}^{\prime}}\xspace} % b'
\providecommand{\PAQbpr}{\ensuremath{\bar{\cmsSymbolFace{b}}^\prime}\xspace} % b'-bar; needs same as anti-t'
\providecommand{\PGg}{\ensuremath{\gamma}\xspace} % gamma
\providecommand{\PKzS}{\ensuremath{\cmsSymbolFace{K}^0_\cmsSymbolFace{S}}\xspace} % K short
\providecommand{\PBs}{\ensuremath{\cmsSymbolFace{B}_\cmsSymbolFace{s}}\xspace} % B sub s
\providecommand{\PSQu}{\ensuremath{\widetilde{\cmsSymbolFace{u}}}\xspace}
\providecommand{\PSQd}{\ensuremath{\widetilde{\cmsSymbolFace{d}}}\xspace}
\providecommand{\PSQc}{\ensuremath{\widetilde{\cmsSymbolFace{c}}}\xspace}
\providecommand{\PSQs}{\ensuremath{\widetilde{\cmsSymbolFace{s}}}\xspace}
\providecommand{\PSQt}{\ensuremath{\widetilde{\cmsSymbolFace{t}}}\xspace} % stop
\providecommand{\PSQb}{\ensuremath{\widetilde{\cmsSymbolFace{b}}}\xspace}
\providecommand{\PASQt}{\ensuremath{\overline{\widetilde{\cmsSymbolFace{t}}}}\xspace} % anti stop
\providecommand{\PASQb}{\ensuremath{\overline{\widetilde{\cmsSymbolFace{b}}}}\xspace} % anti sbottom
\providecommand{\PSGt}{\ensuremath{\widetilde{\tau}}\xspace} % stau
\providecommand{\PZ}{\ensuremath{\cmsSymbolFace{Z}}\xspace} % may have some confusion with the \xspace...
\providecommand{\PZpr}{\ensuremath{\cmsSymbolFace{Z}'}\xspace} % plain Z' using prime
% \renewcommand{\PWpr}{\ensuremath{\cmsSymbolFace{W}'}\xspace} % use prime like pennames2
\providecommand{\PWmp}{\ensuremath{\cmsSymbolFace{W}^\mp}\xspace}
\providecommand{\PWpm}{\ensuremath{\cmsSymbolFace{W}^\pm}\xspace}
\providecommand{\PDstp}{\ensuremath{\cmsSymbolFace{D}^{\ast+}}\xspace}
\providecommand{\PDstm}{\ensuremath{\cmsSymbolFace{D}^{\ast-}}\xspace}
\providecommand{\PGn}{\ensuremath{\nu}\xspace} % generic neutrino
\providecommand{\PAGn}{\ensuremath{\overline{\nu}}\xspace} % generic neutrino
\providecommand{\PSQtDo}{\ensuremath{\widetilde{\cmsSymbolFace{t}}_1}\xspace}
\providecommand{\PSQtDt}{\ensuremath{\widetilde{\cmsSymbolFace{t}}_2}\xspace}
\providecommand{\PQt}{\ensuremath{\cmsSymbolFace{t}}\xspace} % t
\providecommand{\PAQt}{\ensuremath{\overline{\cmsSymbolFace{t}}}\xspace} %
\providecommand{\PQb}{\ensuremath{\cmsSymbolFace{b}}\xspace} % b
\providecommand{\PAQb}{\ensuremath{\overline{\cmsSymbolFace{b}}}\xspace} %
\providecommand{\PGm}{\ensuremath{\mu}\xspace} % muon
\providecommand{\PGmm}{\ensuremath{\mu^-}\xspace} % muon
\providecommand{\PGmp}{\ensuremath{\mu^+}\xspace} % muon
\providecommand{\PGmpm}{\ensuremath{\mu^\pm}\xspace} % muon
\providecommand{\PGt}{\ensuremath{\tau}\xspace} % tau
\providecommand{\PAGt}{\ensuremath{\overline{\tau}}\xspace} % anti-tau
\providecommand{\PGpz}{\ensuremath{\pi^0}\xspace}
\providecommand{\PQq}{\ensuremath{\cmsSymbolFace{q}}\xspace} % quark (generic)
\providecommand{\PQd}{\ensuremath{\cmsSymbolFace{d}}\xspace} % down quark
\providecommand{\PQu}{\ensuremath{\cmsSymbolFace{u}}\xspace} % up quark
\providecommand{\PQs}{\ensuremath{\cmsSymbolFace{s}}\xspace} % top quark
\providecommand{\PQc}{\ensuremath{\cmsSymbolFace{c}}\xspace} % top quark
\providecommand{\PAQq}{\ensuremath{\overline{\cmsSymbolFace{q}}}\xspace} % quark (generic)
\providecommand{\PAQd}{\ensuremath{\overline{\cmsSymbolFace{d}}}\xspace} % down quark
\providecommand{\PAQu}{\ensuremath{\overline{\cmsSymbolFace{u}}}\xspace} % up quark
\providecommand{\PAQs}{\ensuremath{\overline{\cmsSymbolFace{s}}}\xspace} % top quark
\providecommand{\PAQc}{\ensuremath{\overline{\cmsSymbolFace{c}}}\xspace} % top quark
\providecommand{\PGne}{\ensuremath{\nu_\cmsSymbolFace{e}}\xspace} % electron neutrino
\providecommand{\PAGne}{\ensuremath{\overline{\nu}_\cmsSymbolFace{e}}\xspace} % anti-electron neutrino
\providecommand{\PGnGm}{\ensuremath{\nu_\PGm}\xspace} % muon neutrino
\providecommand{\PAGnGm}{\ensuremath{\overline{\nu}_\PGm}\xspace} % anti-muon neutrino
\providecommand{\PGnGt}{\ensuremath{\nu_\PGt}\xspace} % tau neutrino
\providecommand{\PAGnGt}{\ensuremath{\overline{\nu}_\PGt}\xspace} % anti-tau neutrino
\providecommand{\PAp}{\ensuremath{\overline{\cmsSymbolFace{p}}}\xspace} % anti-proton
\providecommand{\PAn}{\ensuremath{\overline{\cmsSymbolFace{n}}}\xspace} % anti-neutron
\providecommand{\PGc}{\ensuremath{\chi}\xspace} % chi (charm, but also SUSY)
\providecommand{\PGcc}{\ensuremath{\chi_{\PQc}}\xspace}
\providecommand{\PGcb}{\ensuremath{\chi_{\PQb}}\xspace}
\providecommand{\PDz}{\ensuremath{\cmsSymbolFace{D}^0}\xspace} % D0 meson
\providecommand{\PADz}{\ensuremath{\overline{\cmsSymbolFace{D}}^0}\xspace} % anti-D0 meson
\providecommand{\PAD}{\ensuremath{\overline{\cmsSymbolFace{D}}}\xspace} % anti-D meson
\providecommand{\PAK}{\ensuremath{\overline{\cmsSymbolFace{K}}}\xspace} % anti-K meson
\providecommand{\PAKz}{\ensuremath{\overline{\cmsSymbolFace{K}}^0}\xspace} % anti-K0 meson
\providecommand{\PABz}{\ensuremath{\overline{\cmsSymbolFace{B}}^0}\xspace} % anti-B0 meson
\providecommand{\PGLb}{\ensuremath{\Lambda_\cmsSymbolFace{b}}\xspace} % Lambda b

% our extensions for pennames2
\providecommand{\Pepm}{\ensuremath{\cmsSymbolFace{e}^\pm}\xspace}
\providecommand{\Pemp}{\ensuremath{\cmsSymbolFace{e}^\mp}\xspace}
\providecommand{\PGmpm}{\ensuremath{\mu^\pm}\xspace}
\providecommand{\PGmmp}{\ensuremath{\mu^\mp}\xspace} % not available in pennames2, AFAIK
% for APS style tables
\ifthenelse{\boolean{cms@external}}{%
 \newenvironment{scotch}[1]{\protect\centering\ruledtabular\tabular{#1}}{\endtabular\endruledtabular}
}{
 \newenvironment{scotch}[1]{\protect\centering\tabular{#1}\hline\hline}{\hline\endtabular}
}
% Other
\newcommand{\MD}{\ensuremath{{M_\mathrm{D}}}\xspace}% ED mass
\newcommand{\Mpl}{\ensuremath{{M_\mathrm{Pl}}}\xspace}% Planck mass
\newcommand{\Rinv} {\ensuremath{{R}^{-1}}\xspace}

% SM (still to be classified)

\newcommand{\AFB}{\ensuremath{A_\text{FB}}\xspace}
\newcommand{\wangle}{\ensuremath{\sin^{2}\theta_{\text{eff}}^\text{lept}(M^2_{\Z})}\xspace}
\newcommand{\stat}{\ensuremath{\,\text{(stat)}}\xspace}
\newcommand{\syst}{\ensuremath{\,\text{(syst)}}\xspace}
\newcommand{\thy}{\ensuremath{\,\text{(theo)}}\xspace}
\newcommand{\lum}{\ensuremath{\,\text{(lumi)}}\xspace}
\newcommand{\kt}{\ensuremath{k_{\mathrm{T}}}\xspace}

\newcommand{\BC}{\ensuremath{\cmsSymbolFace{B_{c}}}\xspace}
\newcommand{\bbarc}{\ensuremath{\PQb\PAQc}\xspace}
\newcommand{\bbbar}{\ensuremath{\PQb\PAQb}\xspace}
\newcommand{\ccbar}{\ensuremath{\PQc\PAQc}\xspace}
\newcommand{\qqbar}{\ensuremath{\PQq\PAQq}\xspace}
\newcommand{\bspsiphi}{\ensuremath{\cmsSymbolFace{B_s} \to \JPsi\, \phi}\xspace}
\newcommand{\EE}{\ensuremath{\Pep\Pem}\xspace}
\newcommand{\MM}{\ensuremath{\Pgmp\Pgmm}\xspace}
\newcommand{\TT}{\ensuremath{\Pgt^{+}\Pgt^{-}}\xspace}

%%%  E-gamma definitions
\newcommand{\HGG}{\ensuremath{\cmsSymbolFace{H}\to\gamma\gamma}}
\newcommand{\GAMJET}{\ensuremath{\gamma + \text{jet}}}
\newcommand{\PPTOJETS}{\ensuremath{\Pp\Pp\to\text{jets}}}
\newcommand{\PPTOGG}{\ensuremath{\Pp\Pp\to\gamma\gamma}}
\newcommand{\PPTOGAMJET}{\ensuremath{\Pp\Pp\to\gamma + \text{jet}}}
\newcommand{\MH}{\ensuremath{M_{\PH}}}
\newcommand{\RNINE}{\ensuremath{R_\mathrm{9}}}
\newcommand{\DR}{\ensuremath{\Delta R}}





%%%%%%
% From Albert
%

\newcommand{\ga}{\ensuremath{\gtrsim}}
\newcommand{\la}{\ensuremath{\lesssim}}
%
\newcommand{\swsq}{\ensuremath{\sin^2\theta_\cmsSymbolFace{W}}\xspace}
\newcommand{\cwsq}{\ensuremath{\cos^2\theta_\cmsSymbolFace{W}}\xspace}
\newcommand{\tanb}{\ensuremath{\tan\beta}\xspace}
\newcommand{\tanbsq}{\ensuremath{\tan^{2}\beta}\xspace}
\newcommand{\sidb}{\ensuremath{\sin 2\beta}\xspace}
\newcommand{\alpS}{\ensuremath{\alpha_S}\xspace}
\newcommand{\alpt}{\ensuremath{\tilde{\alpha}}\xspace}

\newcommand{\QL}{\ensuremath{\cmsSymbolFace{Q}_\cmsSymbolFace{L}}\xspace}
\newcommand{\sQ}{\ensuremath{\widetilde{\cmsSymbolFace{Q}}}\xspace}
\newcommand{\sQL}{\ensuremath{\widetilde{\cmsSymbolFace{Q}}_\cmsSymbolFace{L}}\xspace}
\newcommand{\ULC}{\ensuremath{\cmsSymbolFace{U}_\cmsSymbolFace{L}^\cmsSymbolFace{C}}\xspace}
\newcommand{\sUC}{\ensuremath{\widetilde{\cmsSymbolFace{U}}^\cmsSymbolFace{C}}\xspace}
\newcommand{\sULC}{\ensuremath{\widetilde{\cmsSymbolFace{U}}_\cmsSymbolFace{L}^\cmsSymbolFace{C}}\xspace}
\newcommand{\DLC}{\ensuremath{\cmsSymbolFace{D}_\cmsSymbolFace{L}^\cmsSymbolFace{C}}\xspace}
\newcommand{\sDC}{\ensuremath{\widetilde{\cmsSymbolFace{D}}^\cmsSymbolFace{C}}\xspace}
\newcommand{\sDLC}{\ensuremath{\widetilde{\cmsSymbolFace{D}}_\cmsSymbolFace{L}^\cmsSymbolFace{C}}\xspace}
\newcommand{\LL}{\ensuremath{\cmsSymbolFace{L}_\cmsSymbolFace{L}}\xspace}
\newcommand{\sL}{\ensuremath{\widetilde{\cmsSymbolFace{L}}}\xspace}
\newcommand{\sLL}{\ensuremath{\widetilde{\cmsSymbolFace{L}}_\cmsSymbolFace{L}}\xspace}
\newcommand{\ELC}{\ensuremath{\cmsSymbolFace{E}_\cmsSymbolFace{L}^\cmsSymbolFace{C}}\xspace}
\newcommand{\sEC}{\ensuremath{\widetilde{\cmsSymbolFace{E}}^\cmsSymbolFace{C}}\xspace}
\newcommand{\sELC}{\ensuremath{\widetilde{\cmsSymbolFace{E}}_\cmsSymbolFace{L}^\cmsSymbolFace{C}}\xspace}
\newcommand{\sEL}{\ensuremath{\widetilde{\cmsSymbolFace{E}}_\cmsSymbolFace{L}}\xspace}
\newcommand{\sER}{\ensuremath{\widetilde{\cmsSymbolFace{E}}_\cmsSymbolFace{R}}\xspace}
\newcommand{\sFer}{\ensuremath{\widetilde{\cmsSymbolFace{f}}}\xspace}
\newcommand{\sQua}{\ensuremath{\widetilde{\cmsSymbolFace{q}}}\xspace}
\newcommand{\sUp}{\ensuremath{\widetilde{\cmsSymbolFace{u}}}\xspace}
\newcommand{\suL}{\ensuremath{\widetilde{\cmsSymbolFace{u}}_\cmsSymbolFace{L}}\xspace}
\newcommand{\suR}{\ensuremath{\widetilde{\cmsSymbolFace{u}}_\cmsSymbolFace{R}}\xspace}
\newcommand{\sDw}{\ensuremath{\widetilde{\cmsSymbolFace{d}}}\xspace}
\newcommand{\sdL}{\ensuremath{\widetilde{\cmsSymbolFace{d}}_\cmsSymbolFace{L}}\xspace}
\newcommand{\sdR}{\ensuremath{\widetilde{\cmsSymbolFace{d}}_\cmsSymbolFace{R}}\xspace}
\newcommand{\sTop}{\ensuremath{\widetilde{\cmsSymbolFace{t}}}\xspace}
\newcommand{\stL}{\ensuremath{\widetilde{\cmsSymbolFace{t}}_\cmsSymbolFace{L}}\xspace}
\newcommand{\stR}{\ensuremath{\widetilde{\cmsSymbolFace{t}}_\cmsSymbolFace{R}}\xspace}
\newcommand{\stone}{\ensuremath{\widetilde{\cmsSymbolFace{t}}_1}\xspace}
\newcommand{\sttwo}{\ensuremath{\widetilde{\cmsSymbolFace{t}}_2}\xspace}
\newcommand{\sBot}{\ensuremath{\widetilde{\cmsSymbolFace{b}}}\xspace}
\newcommand{\sbL}{\ensuremath{\widetilde{\cmsSymbolFace{b}}_\cmsSymbolFace{L}}\xspace}
\newcommand{\sbR}{\ensuremath{\widetilde{\cmsSymbolFace{b}}_\cmsSymbolFace{R}}\xspace}
\newcommand{\sbone}{\ensuremath{\widetilde{\cmsSymbolFace{b}}_1}\xspace}
\newcommand{\sbtwo}{\ensuremath{\widetilde{\cmsSymbolFace{b}}_2}\xspace}
\newcommand{\sLep}{\ensuremath{\widetilde{\cmsSymbolFace{l}}}\xspace}
\newcommand{\sLepC}{\ensuremath{\widetilde{\cmsSymbolFace{l}}^\cmsSymbolFace{C}}\xspace}
\newcommand{\sEl}{\ensuremath{\widetilde{\cmsSymbolFace{e}}}\xspace}
\newcommand{\sElC}{\ensuremath{\widetilde{\cmsSymbolFace{e}}^\cmsSymbolFace{C}}\xspace}
\newcommand{\seL}{\ensuremath{\widetilde{\cmsSymbolFace{e}}_\cmsSymbolFace{L}}\xspace}
\newcommand{\seR}{\ensuremath{\widetilde{\cmsSymbolFace{e}}_\cmsSymbolFace{R}}\xspace}
\newcommand{\snL}{\ensuremath{\widetilde{\nu}_L}\xspace}
\newcommand{\sMu}{\ensuremath{\widetilde{\mu}}\xspace}
\newcommand{\sNu}{\ensuremath{\widetilde{\nu}}\xspace}
\newcommand{\sTau}{\ensuremath{\widetilde{\tau}}\xspace}
\newcommand{\Glu}{\ensuremath{\cmsSymbolFace{g}}\xspace}
\newcommand{\sGlu}{\ensuremath{\widetilde{\cmsSymbolFace{g}}}\xspace}
\newcommand{\Wpm}{\ensuremath{\cmsSymbolFace{W}^{\pm}}\xspace}
\newcommand{\sWpm}{\ensuremath{\widetilde{\cmsSymbolFace{W}}^{\pm}}\xspace}
\newcommand{\Wz}{\ensuremath{\cmsSymbolFace{W}^{0}}\xspace}
\newcommand{\sWz}{\ensuremath{\widetilde{\cmsSymbolFace{W}}^{0}}\xspace}
\newcommand{\sWino}{\ensuremath{\widetilde{\cmsSymbolFace{W}}}\xspace}
\newcommand{\Bz}{\ensuremath{\cmsSymbolFace{B}^{0}}\xspace}
\newcommand{\sBz}{\ensuremath{\widetilde{\cmsSymbolFace{B}}^{0}}\xspace}
\newcommand{\sBino}{\ensuremath{\widetilde{\cmsSymbolFace{B}}}\xspace}
\newcommand{\Zz}{\ensuremath{\cmsSymbolFace{Z}^{0}}\xspace}
\newcommand{\sZino}{\ensuremath{\widetilde{\cmsSymbolFace{Z}}^{0}}\xspace}
\newcommand{\sGam}{\ensuremath{\widetilde{\gamma}}\xspace}
\newcommand{\chiz}{\ensuremath{\widetilde{\chi}^{0}}\xspace}
\newcommand{\chip}{\ensuremath{\widetilde{\chi}^{+}}\xspace}
\newcommand{\chim}{\ensuremath{\widetilde{\chi}^{-}}\xspace}
\newcommand{\chipm}{\ensuremath{\widetilde{\chi}^{\pm}}\xspace}
\newcommand{\Hone}{\ensuremath{\cmsSymbolFace{H}_\cmsSymbolFace{d}}\xspace}
\newcommand{\sHone}{\ensuremath{\widetilde{\cmsSymbolFace{H}}_\cmsSymbolFace{d}}\xspace}
\newcommand{\Htwo}{\ensuremath{\cmsSymbolFace{H}_\cmsSymbolFace{u}}\xspace}
\newcommand{\sHtwo}{\ensuremath{\widetilde{\cmsSymbolFace{H}}_\cmsSymbolFace{u}}\xspace}
\newcommand{\sHig}{\ensuremath{\widetilde{\cmsSymbolFace{H}}}\xspace}
\newcommand{\sHa}{\ensuremath{\widetilde{\cmsSymbolFace{H}}_\cmsSymbolFace{a}}\xspace}
\newcommand{\sHb}{\ensuremath{\widetilde{\cmsSymbolFace{H}}_\cmsSymbolFace{b}}\xspace}
\newcommand{\sHpm}{\ensuremath{\widetilde{\cmsSymbolFace{H}}^{\pm}}\xspace}
\newcommand{\hz}{\ensuremath{\cmsSymbolFace{h}^{0}}\xspace}
\newcommand{\Hz}{\ensuremath{\cmsSymbolFace{H}^{0}}\xspace}
\newcommand{\Az}{\ensuremath{\cmsSymbolFace{A}^{0}}\xspace}
\newcommand{\Hpm}{\ensuremath{\cmsSymbolFace{H}^{\pm}}\xspace}
\newcommand{\sGra}{\ensuremath{\widetilde{\cmsSymbolFace{G}}}\xspace}
%
\newcommand{\mtil}{\ensuremath{\widetilde{m}}\xspace}
%
\newcommand{\rpv}{\ensuremath{\rlap{\kern.2em/}R}\xspace}
\newcommand{\LLE}{\ensuremath{LL\bar{E}}\xspace}
\newcommand{\LQD}{\ensuremath{LQ\bar{D}}\xspace}
\newcommand{\UDD}{\ensuremath{\overline{UDD}}\xspace}
\newcommand{\Lam}{\ensuremath{\lambda}\xspace}
\newcommand{\Lamp}{\ensuremath{\lambda'}\xspace}
\newcommand{\Lampp}{\ensuremath{\lambda''}\xspace}
%
\newcommand{\spinbd}[2]{\ensuremath{\bar{#1}_{\dot{#2}}}\xspace}

\endinput







%% add line numbers
%\usepackage[mathlines]{lineno} % add option pagewise for new line number per page
%\linenumbers

\usepackage{ifthen}
\newboolean{bdraft}
\setboolean{bdraft}{true} %%set comments and lipsums on and off
\usepackage{mathtools}
% define some colors
\usepackage[dvipsnames]{xcolor}
\definecolor{lightblue}{rgb}{0.85,0.85,0.92}
\definecolor{gray}{gray}{0.6}
\usepackage{color}
\definecolor{RWTHblue}{RGB}{0,84,159}%RWTH blau
\definecolor{RWTHlightblue}{RGB}{142,186,229}%RWTH hellblau
\definecolor{darkblue}{rgb}{0,0,0.5}

\usepackage{multirow}

\usepackage{placeins}

\usepackage[Glenn]{fncychap}
% \usepackage[Lenny]{fncychap}

\usepackage{minitoc}
\dominitoc[n]
\setcounter{minitocdepth}{1}



% lorem ipsum blind text
%\usepackage{lipsum}
%\newcommand{\lorem}{ \ifthenelse{\boolean{bdraft}} {\textcolor{lightblue}{\lipsum}} {} }
%\setlipsumdefault{1}

% wider lines in tables and arrays
\renewcommand{\arraystretch}{1.1}

% manage "ToDo"s in text
% \usepackage{todo}
\newcommand{\todo}[1]{{\textcolor{red}{TODO \today: #1}}}

% define comments
\newcommand{\comment}[1]{ \ifthenelse{\boolean{bdraft}} {\textcolor{RedOrange}{\{#1\}}} {} }

% Give numbers to deeper levels, and show them in the TOC
%\ifthenelse{\boolean{bdraft}} {\setcounter{tocdepth}{4}} {}
%\ifthenelse{\boolean{bdraft}} {\setcounter{secnumdepth}{4}} {}


% width of pictures (if 2 pictures next to each other)
\newcommand{\pairwidth}{.481\textwidth}

\DeclarePairedDelimiter\bra{\langle}{\rvert}
\DeclarePairedDelimiter\ket{\lvert}{\rangle}
\DeclarePairedDelimiterX\braket[2]{\langle}{\rangle}{#1 \delimsize\vert #2}

% format captions
\usepackage[margin=10pt,skip=8pt, format=plain]{caption}
\KOMAoption{captions}{tableheading, bottombeside}

% configure default position of figures and tables
%\makeatletter
%\renewcommand{\fps@figure}{htbp}
%\renewcommand{\fps@table}{htbp}
%\makeatother

% make tables look nicer
\usepackage{booktabs}

% definition of particle names
\usepackage{general/pennames-pazo}
%\usepackage{hepnames}  % nice particle names, incompatible with mathpazo math fonts

% bibliography support
\usepackage[numbers,sort&compress]{natbib}

%%Feynman graphs
%\usepackage[compat=1.1.0]{tikz-feynman}
\usepackage{feynman}

%\usepackage{feynmp}
%% Automize calls to mpost in TeXnicCenter
%% see http://latex-community.org/forum/viewtopic.php?f=31&t=16193
%\DeclareGraphicsRule{*}{mps}{*}{}
%\makeatletter
%\def\endfmffile{%
%\fmfcmd{\p@rcent\space the end.^^J%
%end.^^J%
%endinput;}%
%\if@fmfio
%\immediate\closeout\@outfmf
%\fi
%\ifnum\pdfshellescape=\@ne
%\immediate\write18{mpost \thefmffile}%
%\fi}
%\makeatother

%links within document
\usepackage[%
 colorlinks, % verwende farbige Links
 linkcolor=black, % Linkfarbe ist RWTH blau
 % citecolor=RWTHlightblue, % Zitatfarbe ist RWTH blau
 citecolor=black, % Zitatfarbe ist RWTH blau
 bookmarks, % erstelle Bookmarks der Links
 bookmarksopen, % Bookmarks werden beim Öffnen des Dokumentes ebenfalls geöffnet
 bookmarksopenlevel=2,
 % urlcolor=RWTHblue, % Hyperlinks sind RWTH blau
 urlcolor=black, % Hyperlinks sind RWTH blau
 bookmarksnumbered, % Bookmarks sind nummeriert
 pdfborder={0 0 0},
 plainpages=false,
 pdfpagelabels,
 % draft  % Draft-Version
 final  % Endversion
]{hyperref}

%\usepackage[
%bookmarksnumbered,
%pdftitle={\titleDocument}%,
%%hyperfootenotes=false
%]{hyperref}

%\hypersetup{%
%%plainpages=false,
%%pdfpagemode=Normal,%Keine Navigatorspalte
%%pdfview=FitH,%Standard-View f�r Link
%%pdfstartview=FitH,%Start-Ansicht FitH,FitV,...
%%pdfpagelayout=TwoColumnRight,%OneColumn,TwoColumnLeft,TwoColumnRight,SinglePage
%%colorlinks=true, % false: boxed links; true: colored links
%%bookmarksopen=true,
%%bookmarksnumbered=true,
%%bookmarksopenlevel=2,
%%pdfmenubar=true,
%%pdfwindowui=true,
%%pdffitwindow=true,
%linkcolor=darkblue,
%%linkcolor=black,
%%linkbordercolor=false,%Rahmenfarbe um Links (1 0 0)Leerzeichen wichtig(R G B)
%citecolor=darkblue,
%%	citecolor=black,
%urlcolor=darkblue,
%filecolor=darkblue
%}

% create glossary
% http://ftp.uni-erlangen.de/ctan/macros/latex/contrib/glossaries/glossaries-user.pdf
% first try of options which are not necessarily optimal
% 'acronym' to obtain list of acronyms independent from glossary
% 'acronym' and 'nomain' to obtain only list of acronyms, without glossary
% 'xindy' uses a perl script to properly sort entries, including e.g. greek letters
%\usepackage[xindy,nomain,acronym,toc]{glossaries}
% set style of glossary
% here: use 2 columns, as we expect short entries (mainly acronyms)
%\usepackage{glossary-mcols}
%\setglossarystyle{mcolindex}
% set style of acronyms
%\setacronymstyle{long-short}
%\makeglossaries  % ensure glossary files are created
