%%% General settings

% \input{general/ptdr-definitions} 	% official CMS definitions
\input{general/my-definitions}		% personal definitions

%%%%%%%%%%%%%%%  Title page %%%%%%%%%%%%%%%%%%%%%%%%
%\documentclass[11pt,twoside, openright, a4paper, pdftex, tdr]{new-cms-tdr}
\documentclass[11pt,twoside, openright, a4paper, pdftex, tdr,chapterprefix=false]{new-cms-tdr}
%\usepackage[left=29mm,right=25mm,top=25mm,bottom=14mm,includeheadfoot]{geometry}%includehead
% \usepackage[left=3cm,right=3cm,top=3cm,bottom=3cm,includeheadfoot]{geometry}%includehead
% \usepackage[left=3cm,right=2cm,top=3cm,bottom=3cm,includeheadfoot]{geometry}%includehead
% \usepackage[left=3cm,right=2cm,top=3cm,bottom=3cm,includehead]{geometry}%includehead
% \usepackage[left=3cm,right=2cm,top=3cm,bottom=3cm]{geometry}%includehead

%\usepackage{geometry} % see geometry.pdf on how to lay out the page. There's lots.
%\geometry{a4paper} % or letter or a5paper or ... etc
% \geometry{landscape} % rotated page geometry

\usepackage[latin1]{inputenc}

\usepackage{general/ptdr-definitions} 	% official CMS definitions


%%MEINE INCLUDES
\usepackage{subfigure}
% \usepackage{amsmath}
\usepackage{graphics}
\usepackage{graphicx}
%\usepackage{caption}
%\usepackage{subcaption}
\usepackage{lmodern}
% \usepackage[ngerman]{babel}
\usepackage[english]{babel}
\usepackage{amssymb}
%\usepackage{hyperref}
%\usepackage{amssymb,amsmath,hyperref}
%\usepackage{svg}

\usepackage{siunitx}
%\newcommand{\pT}{\ensuremath{p_\perp}}
%\newcommand{\pT}{\ensuremath{p_T}}
%\newcommand{\Et}{\ensuremath{E_\perp}}
%\newcommand{\dpt}{\ensuremath{\delta \pT/\pT}}
%\newcommand{\dE}{\ensuremath{\delta E/E}}
%\newcommand{\nm}{\nano\meter}
%\newcommand{\kg}{\kilo\gram}
%\newcommand{\fb}{\femto\barn}
%\newcommand{\fbi}{\ensuremath{\fb^{-1}}}
%
%\newcommand{\pbi}{\ensuremath{\pb^{-1}}}
%\newcommand{\nb}{\nano\barn}
%\newcommand{\nbi}{\ensuremath{\nb^{-1}}}
%\include{General/ptdr-definitions}






%% add line numbers
%\usepackage[mathlines]{lineno} % add option pagewise for new line number per page
%\linenumbers

% \usepackage{ifthen}
\newboolean{bdraft}
\setboolean{bdraft}{true} %%set comments and lipsums on and off
\usepackage{mathtools}
% define some colors
\usepackage[dvipsnames]{xcolor}
\definecolor{lightblue}{rgb}{0.85,0.85,0.92}
\definecolor{gray}{gray}{0.6}
\usepackage{color}
\definecolor{RWTHblue}{RGB}{0,84,159}%RWTH blau
\definecolor{RWTHlightblue}{RGB}{142,186,229}%RWTH hellblau
\definecolor{darkblue}{rgb}{0,0,0.5}

\usepackage{multirow}

\usepackage{placeins}

% \usepackage[Glenn]{fncychap}%this
\usepackage[Bjornstrup]{fncychap}%this
% \usepackage[Lenny]{fncychap}

% \ChNumVar{\Huge\fontfamily{put}\selectfont\color{black}}
\ChNumVar{\fontsize{45pt}{30pt}\selectfont\color{black}}

\usepackage{pdfpages}

\usepackage{minitoc}
\dominitoc[n]
\setcounter{minitocdepth}{1}
\mtcsetfeature{minitoc}{before}{\vspace{-1.5cm}}
\mtcsetfeature{minitoc}{after}{\vspace{.5cm}}


% lorem ipsum blind text
%\usepackage{lipsum}
%\newcommand{\lorem}{ \ifthenelse{\boolean{bdraft}} {\textcolor{lightblue}{\lipsum}} {} }
%\setlipsumdefault{1}

% wider lines in tables and arrays
\renewcommand{\arraystretch}{1.1}

% manage "ToDo"s in text
% \usepackage{todo}
\newcommand{\todo}[1]{{\textcolor{red}{TODO \today: #1}}}

% define comments
\newcommand{\comment}[1]{ \ifthenelse{\boolean{bdraft}} {\textcolor{RedOrange}{\{#1\}}} {} }

% Give numbers to deeper levels, and show them in the TOC
%\ifthenelse{\boolean{bdraft}} {\setcounter{tocdepth}{4}} {}
%\ifthenelse{\boolean{bdraft}} {\setcounter{secnumdepth}{4}} {}


% width of pictures (if 2 pictures next to each other)
\newcommand{\pairwidth}{.481\textwidth}

\DeclarePairedDelimiter\bra{\langle}{\rvert}
\DeclarePairedDelimiter\ket{\lvert}{\rangle}
\DeclarePairedDelimiterX\braket[2]{\langle}{\rangle}{#1 \delimsize\vert #2}

% format captions
% \usepackage[margin=10pt,skip=8pt, format=plain]{caption}
\usepackage[margin=10pt,skip=8pt, format=hang]{caption}
\KOMAoption{captions}{tableheading, bottombeside}

% configure default position of figures and tables
%\makeatletter
%\renewcommand{\fps@figure}{htbp}
%\renewcommand{\fps@table}{htbp}
%\makeatother

% make tables look nicer
\usepackage{booktabs}

% definition of particle names
\usepackage{general/pennames-pazo}
%\usepackage{hepnames}  % nice particle names, incompatible with mathpazo math fonts

% bibliography support
\usepackage[numbers,sort&compress]{natbib}
\usepackage{amsmath}

%%Feynman graphs
%\usepackage[compat=1.1.0]{tikz-feynman}
\usepackage{feynman}





% Disable single lines at the start of a paragraph (Schusterjungen)

\clubpenalty = 10000

% Disable single lines at the end of a paragraph (Hurenkinder)

\widowpenalty = 10000

\displaywidowpenalty = 10000




%\usepackage{feynmp}
%% Automize calls to mpost in TeXnicCenter
%% see http://latex-community.org/forum/viewtopic.php?f=31&t=16193
%\DeclareGraphicsRule{*}{mps}{*}{}
%\makeatletter
%\def\endfmffile{%
%\fmfcmd{\p@rcent\space the end.^^J%
%end.^^J%
%endinput;}%
%\if@fmfio
%\immediate\closeout\@outfmf
%\fi
%\ifnum\pdfshellescape=\@ne
%\immediate\write18{mpost \thefmffile}%
%\fi}
%\makeatother

%links within document
\usepackage[%
 colorlinks, % verwende farbige Links
 linkcolor=black, % Linkfarbe ist RWTH blau
 % citecolor=RWTHlightblue, % Zitatfarbe ist RWTH blau
 citecolor=black, % Zitatfarbe ist RWTH blau
 bookmarks, % erstelle Bookmarks der Links
 bookmarksopen, % Bookmarks werden beim Öffnen des Dokumentes ebenfalls geöffnet
 bookmarksopenlevel=2,
 % urlcolor=RWTHblue, % Hyperlinks sind RWTH blau
 urlcolor=black, % Hyperlinks sind RWTH blau
 bookmarksnumbered, % Bookmarks sind nummeriert
 pdfborder={0 0 0},
 plainpages=false,
 pdfpagelabels,
 % draft  % Draft-Version
 final  % Endversion
]{hyperref}

%\usepackage[
%bookmarksnumbered,
%pdftitle={\titleDocument}%,
%%hyperfootenotes=false
%]{hyperref}

%\hypersetup{%
%%plainpages=false,
%%pdfpagemode=Normal,%Keine Navigatorspalte
%%pdfview=FitH,%Standard-View f�r Link
%%pdfstartview=FitH,%Start-Ansicht FitH,FitV,...
%%pdfpagelayout=TwoColumnRight,%OneColumn,TwoColumnLeft,TwoColumnRight,SinglePage
%%colorlinks=true, % false: boxed links; true: colored links
%%bookmarksopen=true,
%%bookmarksnumbered=true,
%%bookmarksopenlevel=2,
%%pdfmenubar=true,
%%pdfwindowui=true,
%%pdffitwindow=true,
%linkcolor=darkblue,
%%linkcolor=black,
%%linkbordercolor=false,%Rahmenfarbe um Links (1 0 0)Leerzeichen wichtig(R G B)
%citecolor=darkblue,
%%	citecolor=black,
%urlcolor=darkblue,
%filecolor=darkblue
%}

% create glossary
% http://ftp.uni-erlangen.de/ctan/macros/latex/contrib/glossaries/glossaries-user.pdf
% first try of options which are not necessarily optimal
% 'acronym' to obtain list of acronyms independent from glossary
% 'acronym' and 'nomain' to obtain only list of acronyms, without glossary
% 'xindy' uses a perl script to properly sort entries, including e.g. greek letters
%\usepackage[xindy,nomain,acronym,toc]{glossaries}
% set style of glossary
% here: use 2 columns, as we expect short entries (mainly acronyms)
%\usepackage{glossary-mcols}
%\setglossarystyle{mcolindex}
% set style of acronyms
%\setacronymstyle{long-short}
%\makeglossaries  % ensure glossary files are created
