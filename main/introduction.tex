\chapter{Introduction}\label{chap:introduction}

\section{System of units}\label{sec:units}

For simplicity, the unit system commonly used in particle physics is the natural unit system~\cite{UnitSystem}. In natural units, the reduced Planck constant $\hbar$ and the speed of light $c$ are set to unity:
\begin{equation}
  \hbar=c=1
\end{equation}
The most frequently used observables in particle physics are energy, momentum, and mass. They are given in $\GeV$ in the natural unit system. For other variables, such as length and time, the metric unit system is used. Cross sections are given in barn $\left(1\barn=10^{-28}\m^2\right)$. Integrated luminosities are therefore given in $\binv$.

\section{The standard model of particle physics}\label{sec:SM}

The standard model of particle physics (SM) is gauge theory describing three of the four fundamental forces, namely the electromagentic, weak, and strong interaction~\cite{SM}. The gravitional force is described in general relativity~\cite{Einstein}.\\
All fundamental particles can be devided in two subclasses: Particles of integer-spin, called bosons, and particles of half-integer spin, called fermions.\\\\
The SM is based on the symmetry group $SU(3)\otimes SU(2)\otimes U(1)$. The interactions are characterized via the exchange of spin-1 gauge fields, which are the bosons. In the case of the strong force these are 8 massles gluons, which couple to the color charge. The mediator of the electromagnetic interaction is the massles photon, coupling to the electric charge of particles. For the weak interaction these are the three massive bosons $\PWpm$ and $\PZ$, which couple to weak charge.\\\\

While the bosons describe the mediation of the fundamental forces, the matter content is given by the fermions. Fermions are divided into two subclasses, called quarks and leptons. Leptons take part only in the electroweak interaction, while quarks carry also a color charge and can therefore interact via the strong force. There exist three generations of fermions, which include each two lepton and two quark flavours. The quark flavours are namely the down, up, strange, charm, bottom, and top quarks, while the lepton flavours are made up of three electrically charged particles, the electron ($\Pe$), the muon ($\PGm$), and the tau lepton ($\PGt$), and three electric neutral leptons, called neutrinos ($\Pgne,\Pgngm,\Pgngt$). The latter are assigned the names of the charged leptons of the same generation. Of the quarks, there are up-type quarks carrying the electric charge of $+\frac{2}{3}e$, and down-type quarks carrying the electric charge of $-\frac{1}{3}e$.\\
An illustration of all partiles with its properties can be seen in \refFig{fig:SM}.
For each particle a corresponding anti-particle exists with same mass and inversed quantum numbers. Troughout this thesis particles and antiparticles will be treated the same way and will be labeled with the name of the particle.\\

\begin{figure}[!htpb]
\centering
  \includegraphics[width=0.8\textwidth]{figures/general/SM}
  \label{fig:SM}
  \caption{\todo{caption Werte aus PDG}}
\end{figure}


The strong interaction between quarks and gluons is described in the quantum field theory of quantum chromodynamics (QCD). The corresponding mediators of the non-abelian gauge group $SU(2)_C$ are the eight gluons, which carry each the color-charge $C$ of an anti-color and color, leading to the self coupling of gluons. Due to the confinement of quarks~\cite{Confinement}, quark-antiquark pairs will be produced out of the vacuum, if particles with color charge will be separated, since the potential energy density of the strong force has constant constant terms and the potential energy rises with increasing distance. The same principle leads to the existance of only color-neutral bound states of two (mesons), or three (baryons) quarks, called hadrons.
\\\\
The electromagnetic and weak force can be unified in the electroweak theory to obtain the electroweak interaction~\cite{Weinberg,Weinberg2,Salam,Glashow}, represented by the gauge group $ SU(2)_L\otimes U(1)_Y$. The indices $L$ and $Y$ indicate that the weak isospin $T$ couples only to lefthanded $SU(2)_L$ doublets of fermions, while the righthanded $SU(2)_L$ singlets carry no isospin, and $Y$ is the hypercharge. The three mediators of the $SU(2)_L$ group are the $W^1,W^2$, and $W^3$ bosons, and the gauge boson of the $U(1)_Y$ group is the $B^0$ boson.
Due to the spontaneous symmetry breaking in the electroweak unification, these four bosons mix to the observed $\PWpm$ and $\PZ$ boson, and the photon $\PGg$:
\begin{equation}
  \left(
  \begin{matrix}
    \PGg \\
    \PZ
  \end{matrix}
  \right)
  =
  \left(\begin{matrix}
  \cos(\theta_W) & \sin(\theta_W)\\
  -\sin(\theta_W) & \cos(\theta_W)
\end{matrix}\right)
\cdot
\left(
\begin{matrix}
  B \\
  W^3
\end{matrix}
\right)
\end{equation}

\begin{equation}
  \PWpm = \frac{1}{\sqrt{2}}\left(W^1\mp i W^2\right)
\end{equation}

\\Because in this theory the gauge bosons are not allowed to have masses, the Higgs mechanism is introduced~\cite{Higgs1,Higgs2,Higgs3}.





\subsection{Indications for physics beyond the standard model}\label{sec:SM_bsm}

\section{Supersymmetry}\label{sec:SUSY}

\subsection{General gauge mediation}\label{sec:GGM}

\subsection{Signal scenarios}\label{sec:SMS}
