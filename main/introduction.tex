\chapter{Introduction}\label{chap:introduction}

\section{System of units}\label{sec:units}

For simplicity, the unit system commonly used in particle physics is the natural unit system~\cite{UnitSystem}. In natural units, the reduced Planck constant $\hbar$ and the speed of light $c$ are set to unity:
\begin{equation}
  \hbar=c=1
\end{equation}
The most frequently used observables in particle physics are energy, momentum, and mass. They are given in $\GeV$ in the natural unit system. For other variables, such as length and time, the metric unit system is used. Cross sections are given in barn $\left(1\barn=10^{-28}\m^2\right)$. Integrated luminosities are therefore given in $\binv$.

\section{The standard model of particle physics}\label{sec:SM}

The standard model of particle physics (SM) is gauge theory describing three of the four fundamental forces, namely the electromagentic, weak, and strong interaction~\cite{SM}. The gravitional force is described in general relativity~\cite{Einstein}.\\
All fundamental particles can be devided in two subclasses: Particles of integer-spin, called bosons, and particles of half-integer spin, called fermions.\\\\
The SM is based on the symmetry group $SU(3)\otimes SU(2)\otimes U(1)$. The interactions are characterized via the exchange of spin-1 gauge fields, which are the bosons. In the case of the strong force these are 8 massles gluons, which couple to the color charge. The mediator of the electromagnetic interaction is the massles photon, coupling to the electric charge of particles. For the weak interaction these are the three massive bosons $\PWpm$ and $\PZ$, which couple to weak charge.\\\\

While the bosons describe the mediation of the fundamental forces, the matter content is given by the fermions. Fermions are divided into two subclasses, called quarks and leptons. Leptons take part only in the electroweak interaction, while quarks carry also a color charge and can therefore interact via the strong force. There exist three generations of fermions, which include each two lepton and two quark flavours. The quark flavours are namely the down, up, strange, charm, bottom, and top quarks, while the lepton flavours are made up of three electrically charged particles, the electron ($\Pe$), the muon ($\PGm$), and the tau lepton ($\PGt$), and three electric neutral leptons, called neutrinos ($\Pgne,\Pgngm,\Pgngt$). The latter are assigned the names of the charged leptons of the same generation. Of the quarks, there are up-type quarks carrying the electric charge of $+\frac{2}{3}e$, and down-type quarks carrying the electric charge of $-\frac{1}{3}e$.\\
An illustration of all partiles with its properties can be seen in \refFig{fig:SM}.
For each particle a corresponding anti-particle exists with same mass and inversed quantum numbers. Troughout this thesis particles and antiparticles will be treated the same way and will be labeled with the name of the particle.\\

\begin{figure}[!htpb]
\centering
  \includegraphics[width=0.8\textwidth]{figures/general/SM}
  \label{fig:SM}
  \caption{\todo{caption Werte aus PDG}}
\end{figure}


The strong interaction between quarks and gluons is described in the quantum field theory of quantum chromodynamics (QCD). The corresponding mediators of the non-abelian gauge group $SU(2)_C$ are the eight gluons, which carry each the color-charge $C$ of an anti-color and color, leading to the self coupling of gluons. Due to the confinement of quarks~\cite{Confinement}, quark-antiquark pairs will be produced out of the vacuum, if particles with color charge will be separated, since the potential energy density of the strong force has constant constant terms and the potential energy rises with increasing distance. The same principle leads to the existance of only color-neutral bound states of two (mesons), or three (baryons) quarks, called hadrons.
\\\\
The electromagnetic and weak force can be unified in the electroweak theory to obtain the electroweak interaction~\cite{Weinberg,Weinberg2,Salam,Glashow}, represented by the gauge group $ SU(2)_L\otimes U(1)_Y$. The indices $L$ and $Y$ indicate that the weak isospin $T$ couples only to lefthanded $SU(2)_L$ doublets of fermions, while the righthanded $SU(2)_L$ singlets carry no isospin, and $Y$ is the hypercharge. The three mediators of the $SU(2)_L$ group are the $W^1,W^2$, and $W^3$ bosons, and the gauge boson of the $U(1)_Y$ group is the $B^0$ boson.
Due to the spontaneous symmetry breaking in the electroweak unification, these four bosons mix to the observed $\PWpm$ and $\PZ$ boson, and the photon $\PGg$:
\begin{equation}
  \left(
  \begin{matrix}
    \PGg \\
    \PZ
  \end{matrix}
  \right)
  =
  \left(\begin{matrix}
  \cos(\theta_W) & \sin(\theta_W)\\
  -\sin(\theta_W) & \cos(\theta_W)
\end{matrix}\right)
\cdot
\left(
\begin{matrix}
  B \\
  W^3
\end{matrix}
\right)
\end{equation}

\begin{equation}
  \PWpm = \frac{1}{\sqrt{2}}\left(W^1\mp i W^2\right)
\end{equation}
The resulting weak interaction is parity violating. The $\PWpm$ bosons only couple to lefthanded fermions, while the neutral Z boson couples to both lefthanded and righthanded particles, but with different strength.\\
\\Because in this theory the gauge bosons are not allowed to have masses, the Higgs mechanism is introduced~\cite{Higgs1,Higgs2,Higgs3}. It predicts a complex scalar doublet Higgs field, which is symmetric but has a non zero vacuum expectation value and is responsable for the spontaneous symmetry breaking of the $ SU(2)_L\otimes U(1)_Y$ gauge group. Since it has four degrees of freedom, but only three are used to give the $\PWpm$ and the Z boson masses, a fourth spin-0 boson is postulated, the Higgs boson. Leptons aquire also masses in the SM via Yukawa couplings with the Higgs field. A spin-0 scalar boson has been observed in proton-proton collisions at the LHC in 2012~\cite{HiggsCMS,HiggsATLAS}, and its mass has been determined to be $125.09\pm0.24\GeV$. This theory earned validation in good agreement with SM predictions~\cite{HiggsPrecise}, and recently also couplings to the top quark~\cite{ttH}, and decays to bottom quarks and tau leptons have been observed~\cite{HiggsTauTau,HiggsBB}.





\subsection{Indications for physics beyond the standard model}\label{sec:SM_bsm}
Since the SM describes all phenomena observed at high energy particle colliders succesfully, other observations indicate that there must exist physics beyond the standard model (BSM).\\
Precise measurements of the cosmic microwave background and theoretical interpretations suggest, that only $4.9\%$ of the universe consist of ordinary matter, while the remainder is dark energy and dark matter~\cite{DarkMatterPlanck}. The exisctence of dark matter is also observed in gravitational lensing effects~\cite{DarkMatterLensing}, and in rotation curves of spiral galaxies~\cite{DarkMatterRotation}. But inside the SM there is no particle that could explain the total amount of dark matter in the universe.\\
It is assumed, that in the early age of the universe there existend the same amount of matter and antimatter. But, today we observe the existence of much more matter than antimatter~\cite{Antimatter,AsymSM}. Conditions, such as $CP$-violation, and baryon number violation, should be fulfilled ~\cite{Sakharov}, so that this discrepancy can be explained. However, there are no known sources of violation effects that are large enough to give rise to such big differences.\\
In the SM neutrinos are assumed to be massles particles, but the observation of neutrino oscillations are only explicable if neutrinos have masses~\cite{NeutrinoMass,PDG}.\\
Also, since the weak and electromagnetic forces can be unified to the electroweak interaction, the unification of all forces in a grand unified theory (GUT) is well motivated. Because the couplings of the forces in the SM do not lead to a unification at very high energies~\cite{PDG}, a possible extenstion of the SM with new particles could explain such a unification of the electroweak and strong interaction. One of those theories is supersymmetry~\cite{SUSYPrimer}.


\section{Supersymmetry}\label{sec:SUSY}

\subsection{General gauge mediation}\label{sec:GGM}

\subsection{Signal scenarios}\label{sec:SMS}
